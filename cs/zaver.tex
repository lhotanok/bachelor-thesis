\chapter*{Závěr}
\addcontentsline{toc}{chapter}{Závěr}

Cílem této bakalářské práce bylo vytvořit aplikaci pro vyhledávání receptů, která agreguje data z~různých zdrojů s~využitím principu propojených dat. Demonstrovali jsme přínosy a~úskalí statických datasetů i~vlastní extrakce dat. Dále jsme zapojili informace z~otevřených znalostních grafů, které nám umožnily nejen rozšířit data o~jednotlivých ingrediencích, ale také objevit nová spojení mezi přísadami a~tranzitivně i~recepty.

V~prvotní fázi vývoje byly zadefinovány funkční i~nefunkční požadavky aplikace, na základě kterých byla navržena architektura a~uživatelské rozhraní. Výběr požadavků byl veden analýzou existujících webových aplikací pro vyhledávání receptů. Naše práce si totiž stanovila cíl propojit užitečné funkce z~osvědčených aplikací a~nabídnout přidanou hodnotu díky jejich zkombinování v~rámci jedné aplikace. Volba klíčových technologií se ukázala jako vhodná pro naplnění požadavků.

Práce se soustředila pouze na získání a~prezentaci existujících dat receptů a~ingrediencí. Přidávání nových dokumentů prostřednictvím uživatelského rozhraní aplikace bude otázkou dalšího vývoje spolu s~registrací uživatelů, ukládání oblíbených receptů, nákupního seznamu a~dalších souvisejících funkcí.

Zkorigování dat z~více zdrojů vyžaduje poměrně velké množství manuální práce, neboť mohou obsahovat řadu odchylek od standardního formátu. Pokud bychom ale měli k~dispozici dostatečnou časovou dotaci a~výpočetní prostředky, aplikace by mohla vyniknout právě velkým množstvím dostupných dat. S~rostoucí datovou sadou bychom byli schopni nabídnout rozmanitější filtrování výsledků nebo také personalizované návrhy receptů.

Aktuální řešení je připraveno pro nasazení a~vývoj dalších funkcionalit. Postup extrakce a~ukládání nových dokumentů je v~co největší míře automatizován pomocí vlastních skriptů, stejně jako definice schématu pro vyhledávání na základě indexů.