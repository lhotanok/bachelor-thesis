%%% Fiktivní kapitola s ukázkami sazby

\chapter{Kontext vývoje}

V~této kapitole si rozebereme různé alternativy datových sad pro naši aplikaci a představíme jejich výhody i~nevýhody vzhledem k~požadavkům aplikace. Dále se podíváme na existující webové stránky s~recepty a~provedeme diskuzi nad jejich funkcemi, možnými vylepšeními a~rozšířeními.

\section{Dostupné datové sady}

V~první fázi analýzy se zaměříme na veřejně dostupná zdrojová data s~recepty, která by mohla posloužit jako podklad pro naši databázi. Jedním z~nejdůležitějších projektů v~této oblasti je \emph{Recipe1M+}, strukturovaný korpus obsahující přes 1 milion receptů a 13 milionů souvisejících obrázků jídla. Aktuálně se jedná o~největší veřejně dostupnou sadu receptů. Dataset je dostupný pouze přihlášeným uživatelům z~ověřené organizace a je povoleno jej využívat pouze pro studium a výzkum. Z~celkového počtu 1 milionu receptů obsahuje 50 000 receptů s~nutričními informacemi \citep{marin2019learning}. V~naší aplikaci plánujeme nutriční hodnoty zahrnout, takže bychom měli k~dispozici 50 000 dokumentů s touto informací. Ostatní data jsou určena přednostně pro strojové zpracování prostřednictvím trénování modelů a~pro finální zobrazení uživateli v~rámci webové aplikace se nezdají být příliš vhodná. Celková velikost datové sady se navíc pohybuje v~řádu stovek gigabytů, což je pro potřeby této práce příliš mnoho vzhledem k~omezené výpočetní kapacitě.

Dalším významným aktérem na poli volně dostupných receptů je iniciativa \emph{Open Recipes}. Autoři projekt definují jako otevřenou databázi záložek s~recepty. Pojem záložky je použit z~důvodu absence instrukcí k~přípravě receptu. Dataset má sloužit pouze k~vyhledání receptu a~pro detailní informace má být uživatel přesměrován na zdroj s~kompletním receptem \citep{open-recipes}. Tento přístup úspěšně využívají některé z~vyhledávačů receptů, např. populární aplikace \emph{SuperCook}. Naše aplikace si ale klade za cíl zpracovat i~stránky s~detaily receptů, ze kterých lze dále pokračovat na detaily ingrediencí s~informacemi ze znalostních grafů. Projekt Open Recipes tedy nebude vhodnou volbou.