%%% Fiktivní kapitola s ukázkami sazby

\chapter{Analýza}

V~této kapitole si rozebereme různé alternativy datových sad pro naši aplikaci a srovnáme jejich výhody i~nevýhody vzhledem k~požadavkům aplikace. Dále se podíváme na existující webové stránky s~recepty a~provedeme diskuzi nad jejich funkcemi, možnými vylepšeními a~rozšířeními.

\section{Požadavky aplikace}

\section{Dostupné datové sady}

V~první fázi analýzy se zaměříme na veřejně dostupná zdrojová data s~recepty, která by mohla posloužit jako podklad pro naši databázi. Jedním z~nejdůležitějších projektů v~této oblasti je \emph{Recipe1M+}, strukturovaný korpus obsahující přes $1$~milion receptů a $13$ milionů souvisejících obrázků jídla. Aktuálně se jedná o~největší veřejně dostupnou sadu receptů. Dataset je dostupný pouze přihlášeným uživatelům z~ověřené organizace a je povoleno jej využívat pouze pro účely studia a výzkumu. Z~celkového počtu $1$~milionu receptů obsahuje $50\,000$ receptů s~nutričními informacemi \citep{marin2019learning}. V~naší aplikaci preferujeme nutriční hodnoty zahrnout, pokud jsou dostupné na zdrojové stránce receptu. Měli bychom tedy k~dispozici $50\,000$ dokumentů s touto informací. Ostatní data jsou určena přednostně pro strojové zpracování prostřednictvím trénování modelů. Celková velikost datové sady se pohybuje v~řádu stovek gigabytů, samotné JSON dokumenty se strukturovanými recepty z adresáře \texttt{layers} se ale vejdou do $2~GiB$, tudíž by byly vhodné pro potřeby této práce limitované omezenou výpočetní kapacitou. Lze odtud využít $1\,029\,720$ receptů obsahujících název, url, ingredience a postup přípravy. Odkazy na ilustrační fotografie jsou u $402\,760$ z těchto receptů. Pro příjemnější uživatelský zážitek se omezíme pouze na recepty s obrázky, takže jsme z datasetu Recipe1M+ schopni použít přibližně $400\,000$ receptů, pokud akceptujeme absenci nutričních hodnot. Bylo by spíše obtížnější z tohoto datasetu identifikovat názvy ingrediencí, neboť jsou suroviny uloženy včetně jejich množství a jednotek měření v rozmanitém formátu.

Dalším významným aktérem na poli volně dostupných receptů je iniciativa \emph{Open Recipes}. Autoři Finkler, Shiflett a Birkebæk projekt představují jako otevřenou databázi záložek s~recepty. Pojem záložky je použit z~důvodu absence instrukcí k~přípravě receptu. Dataset má sloužit pouze k~vyhledání receptu a~pro detailní informace má být uživatel přesměrován na zdroj s~kompletním receptem \citep{open-recipes}. Tohoto přístupu úspěšně využívají některé z~vyhledávačů receptů, např. populární aplikace \emph{SuperCook}. Naše aplikace si ale klade za cíl zpracovat i~stránky s~detaily receptů, ze kterých lze dále pokračovat na detaily ingrediencí s~informacemi ze znalostních grafů. Projekt Open Recipes tedy pro náš scénář nebude vhodnou volbou.

Rozsáhlý dataset \emph{Food.com Recipes and Interactions} s téměř $200\,000$ recepty extrahovanými z webové stránky \emph{Food.com} (původního GeniusKitchen) je publikován na portálu \emph{Kaggle}, který shromažďuje podklady pro strojové učení. Datová sada pokrývá $18$ let interakce uživatelů včetně hodnocení, počtu recenzí i~konkrétních reakcí \citep{shuyang_li_2019}. Kromě základních informací obsahuje také nutriční hodnoty receptů, datum publikování a rovněž normalizovaná jména ingrediencí. Ta byla získána parsováním originálního textu surovin, kvůli čemuž nejsou vždy zcela spolehlivě přesná (např. ve jménech často zůstala jednotka měření z původního textu). Unikátních ingrediencí je k~dispozici kolem $8\,000$, což by měl být dostačující základ pro hledání linků s~entitami otevřených znalostních grafů. Zároveň ve srovnání s~předchozími projekty nabízí nejbohatší informace k jednotlivým receptům. Nevýhodou datasetu je jeho primární určení pro strojové zpracování. Byl vytvořen jako podklad pro generování personalizovaných receptů na základě dřívějších preferencí uživatele \citep{majumder-etal-2019-generating}. Syrová data nejsou zamýšlena pro přímou prezentaci, což se negativně odráží na jejich přesnosti a estetice. Slova jsou občas zařazena do špatných kategorií a~problematický je zejména plně \emph{lowercase} formát textu, ze kterého nejsme schopni zpětně zrekonstruovat originální text z aplikace Food.com. Dataset bychom tedy nemohli použít samostatně, ale jedině s~kombinací vlastní extrakce dat, která by respektovala velikost písma a~lépe se vypořádala s~parsováním jednotlivých kategorií.
Tento problém je poměrně snadno řešitelný díky struktuře stránky Food.com. Z~unikátního id receptu lze jednoduše složit url ve formátu \texttt{www.food.com/recipe/id} a~navíc aplikace podporuje koncept propojených dat, tedy poskytuje recepty ve strukturovaném RDF formátu. Do HTML hlaviček všech dokumentů s~recepty vkládá JSON-LD serializaci dle ontologie \emph{Schema.org}. Z připraveného datasetu bychom tedy mohli využít identifikátory receptů a normalizované ingredience, pro každý recept extrahovat jeho JSON-LD a spojit informace dohromady. Zároveň bychom si ušetřili práci s převáděním receptu do JSON-LD formátu a místo toho mohli použít již předpřipravený soubor a ten vložit do hlavičky dokumentu receptu.

Přímo v oblasti znalostních grafů figuruje 