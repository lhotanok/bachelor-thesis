\chapter{Testování}

\section{Uživatelské testování}

Uživatelské rozhraní a~celkový dojem z~aplikace otestujeme pomocí rozšířené metody System Usability Scale, označované zkráceně jako SUS \citep{sus-test}. Tuto metodu představil John Brooke již v~roce $1986$, stále je ale velmi aktuální, o~čemž svědčí například její zmínka ve více než $1300$ publikacích.

Účastníkům testování nejprve představíme základní funkce, tedy vyhledávání receptů pomocí různých filtrů, zobrazení detailu receptu s~odkazy na rozšířené informace o~ingrediencích a~obsah stránky s~detailem ingredience. Následně jim uložíme několik úkolů, které je přimějí seznámit se s~aplikací více do hloubky. Po splnění úkolů necháme uživatele vyplnit standardizovaný dotazník s~$10$ otázkami a~na základě odpovědí vypočteme koeficient úspěšnosti.

Výsledné skóre dotazníku se pohybuje v rozmezí od $0$ do $100$, nejedná se ale o~procentuální stupnici, jak by se mohlo na první pohled zdát. Důležitá je hranice $68$ bodů. Pokud je překročena, použitelnost systému je hodnocena jako nadprůměrná. Skóre lze dále normalizovat pro přesnější interpretaci výsledků, viz~obrázek \ref{obr04:sus-score-system}. Nám vyšlo na základě odpovědí $4$ respondentů průměrné skóre $93,125$, což lze považovat za velmi dobrý výsledek. Detaily výpočtu a~konkrétní odpovědi z~dotazníku jsou umístěny v~příloze.

\begin{figure}[h!]\centering
\includegraphics[width=140mm]{../img/sus-score-system}
\caption{Interpretace skóre při testování metodou System Usability Score. Adaptováno ze schématu SUS pro UX \citep{sus-adobe}.}
\label{obr04:sus-score-system}
\end{figure}

\subsection{Zadané úkoly}

Uživatelům zadáme následující úlohy, které simulují běžné využívání aplikace:

\begin{enumerate}
    \item Najděte recepty, které obsahují kuřecí prsa, rajčata a~parmazán a~je možné je připravit do $3/4$ hodiny. Poté vyhledávání změňte na recepty, které lze připravit do $1$~a~půl hodiny.
    \item Najděte nejrychlejší veganské recepty vhodné pro přípravu snídaně.
    \item Zadejte libovolné filtry a~projděte výsledky vyhledávání na první a~druhé straně. Vyberte $3$~recepty, které vás nejvíce zaujaly a~zběžně projděte jejich obsah včetně postupu přípravy.
    \item Najděte recepty z~mexické kuchyně s vysokým obsahem bílkovin a~nízkým obsahem sacharidů. Bez otevření detailů zjistěte, z~jakých ingrediencí jsou vyrobeny.
    \item Prozkoumejte ingredience na stránce vybraného receptu a~najděte suroviny ze stejných kategorií.
\end{enumerate}

\subsection{Zpětná vazba}

Nezávisle na SUS dotazníku jsme požádali testující také o~celkovou zpětnou vazbu a~nápady na vylepšení v~rámci dalšího vývoje. Celkový dojem z~používání aplikace byl pozitivní, obdrželi jsme ale řadu připomínek a~návrhů, díky kterým by práce s~aplikací byla pohodlnější a~jednodušší. Výběr nejdůležitějších z~nich, které budou vyřešeny přednostně:

\begin{enumerate}
    \item Při otevřeném receptu nebo ingredienci by měla být v~horní části stránky možnost navigace, aby uživatel nemusel využívat tlačítko \texttt{Zpět} ve webovém prohlížeči, které je zbytečně daleko a~ruší dojem single-page aplikace. Aktuální navigace v~podobě \emph{breadcrumbs} na stránce receptu je pro tento scénář nedostačující, neboť při otevření kategorie \texttt{Recipes} jsou ztraceny původní vyhledávací filtry.
    \item Názvy některých ingrediencí z~našeptávače pro vyhledávání obsahují překlepy nebo gramatické chyby.
    \item Řada kategorií na stránkách ingrediencí je příliš odborná a~do aplikace tohoto typu se spíše nehodí.
    \item Měly by být k~dispozici další jazykové lokalizace. Spoustu anglických termínů ingrediencí a~klíčových slov běžný uživatel nezná, pokud není rodilý mluvčí.
\end{enumerate}

Kromě závěrečného testování probíhaly také průběžné konzultace, na základě kterých byla do aplikace přidána funkce cachování, dále byly opraveny chyby s~nesprávně namapovanými ingrediencemi a~byl zkorigován vzhled aplikace.

\section{Výkonnostní testování}

Zátěžové testy provedeme v~aplikaci pracující přibližně s~$91\,000$ recepty a~té\-měř $800$ ingrediencemi s~rozšiřujícími informacemi. Polovina dokumentů s~recepty pochází z~aplikace Food.com, druhá polovina z~Allrecipes. Při vyhledávání se zobrazuje maximálně $24$ výsledků na jedné straně pro optimalizaci renderování.

Všechny součásti aplikace včetně databáze CouchDB, vyhledávací platformy Solr a~serverové i~klientské části běží na jednom zařízení se stejnou IP adresou. Nemůžeme tedy zcela simulovat reálnou situaci, kdy se do zpracování vyhledávacích dotazů vloží komunikace se vzdáleným zařízením přes HTTP. Všechny API~požadavky totiž vyřizujeme lokálně.

Testovacím zařízením bude notebook Huawei se $4$-jádrovým ~procesorem Intel Core $i5$ $10$. generace a~$16$GB pamětí RAM. Na zařízení je nainstalován operační systém Windows $10$ a hlavní webové prohlížeče, na které cílíme, tedy Google Chrome, Mozilla Firefox a~Microsoft Edge. Testy výkonu provedeme pouze v prohlížeči Google Chrome, u~ostatních dvou pouze zkontrolujeme správné zobrazení aplikace. Využijeme vestavěný nástroj pro diagnostiku výkonu dostupný pod záložkou \texttt{Performance insights}. Pro účely testování vytvoříme produkční verzi aplikace pomocí příkazu \texttt{npm\,run\,build} a~následného \texttt{serve\,-s\,build}.

\subsection{Diagnostika výkonu}

Podíváme se na rychlost získávání dat ze serverové části aplikace, renderování obsahu HTML prostřednictvím modelu DOM, ale také na vizuální stabilitu jednotlivých elementů. Kromě těchto údajů diagnostika odhalila některé déle běžící funkce na hlavním vlákně. U~nich je doporučena obecná optimalizace v~podobě snížení času výpočtu na hlavním vlákně pod hranici $50\,$ms. Celkově jsou naměřené časy v~normě vzhledem k~nefunkčním požadavkům aplikace z~kapitoly o~analýze.

\subsubsection{Vyhledávání receptů}

Testy ukázaly, že obsah DOM pro obrazovku s~vyhledáváním receptů je bez zadání jakýchkoli filtrů načten průměrně za $0,5\,$s. Z toho přibližně $180\,$ms trvá získání receptů ze Solr. Vyžádáme si sice pouze $24$ receptů, chceme ale znát celkový počet pro aktuální vyhledávací dotaz. Kvůli tomu trvá zpracování dotazu na všechny recepty delší dobu než vyhledávání s~omezujícím filtrem nebo více filtry.

Z~filtrovacích dotazů jsme nejprve testovali ingredienci \texttt{Cheese}, pro kterou je k~dispozici téměř $23\,000$ receptů. Zpracování tohoto dotazu trvalo průměrně $115\,$ms, tedy přibližně o třetinu kratší dobu oproti vyhledávání všech receptů. Se zadáváním dalších filtrů klesá čas potřebný pro získání dat až na průměrných $25\,$ms pro dotaz filtrující právě $1$~recept. Tento klesající trend je pro naši aplikaci příznivý, neboť cílíme na specifické filtrování pomocí většího množství kritérií. Nicméně pokud bychom datovou sadu s~recepty výrazně rozšiřovali, museli bychom optimalizovat dotaz na všechny výsledky bez zadaných filtrů. Jeho zpracování by bez optimalizace trvalo příliš dlouho, což by bylo krajně nevhodné, neboť se jedná o~dotaz aktivovaný automaticky při otevření domovské stránky a~tedy první interakci uživatele s~aplikací.

\subsubsection{Detail receptu}

Získání dat receptu na základě id trvá průměrně $10\,$ms. Načtení DOM pak $0,24\,$s. Z~následného vykreslení obvykle připadne nejdelší čas (kolem $0,85\,$s) na popis receptu, pokud je text delší.

\subsubsection{Detail ingredience}

Načtení dokumentu ingredience z~backendu trvá stejně jako u~receptu přibližně $10\,$ms a~načtení DOM asi $0,3\,$s. Maximální čas vykreslení je opět spotřebován renderováním popisu (pro ingredienci \texttt{Beef} se jednalo o~$0,72\,$s).

\subsubsection{Vizuální stabilita}

Celkově by dle výsledků testování bylo potřeba zapracovat na tzv.~\emph{Cumulative Layout Shift} (CLS). Jedná se o~metriku zaměřenou na vizuální stabilitu rozložení jednotlivých elementů v HTML dokumentu. Typicky je ohrožena asynchronními požadavky, na základě kterých se data načítají postupně a~způsobují tak přeskládávání elementů. Je tedy vhodné ještě před načtením dat vytvořit aspoň základní rozložení v~očekávané velikosti, aby se následným přesunům zamezilo nebo byl zmírněn jejich dosah. Často totiž předem nevíme, v~jaké velikosti data obdržíme a~nemůžeme tak přesně odhadnout rozměry příslušného elementu. Nicméně už od skóre $0,25$ je potřeba cílit na zlepšení \citep{cls-metric} a~diagnostický nástroj u~některých z~našich elementů zachytil podstatně vyšší skóre $0,65$. Jedná se například o~sloupec s~nutričními hodnotami a~ingrediencemi na stránce detailu receptu nebo o~společnou patičku stránky.