%%% Hlavní soubor. Zde se definují základní parametry a odkazuje se na ostatní části. %%%

%% Verze pro jednostranný tisk:
% Okraje: levý 40mm, pravý 25mm, horní a dolní 25mm
% (ale pozor, LaTeX si sám přidává 1in)
\documentclass[12pt,a4paper]{report}
\setlength\textwidth{145mm}
\setlength\textheight{247mm}
\setlength\oddsidemargin{15mm}
\setlength\evensidemargin{15mm}
\setlength\topmargin{0mm}
\setlength\headsep{0mm}
\setlength\headheight{0mm}
% \openright zařídí, aby následující text začínal na pravé straně knihy
\let\openright=\clearpage

%% Pokud tiskneme oboustranně:
% \documentclass[12pt,a4paper,twoside,openright]{report}
% \setlength\textwidth{145mm}
% \setlength\textheight{247mm}
% \setlength\oddsidemargin{14.2mm}
% \setlength\evensidemargin{0mm}
% \setlength\topmargin{0mm}
% \setlength\headsep{0mm}
% \setlength\headheight{0mm}
% \let\openright=\cleardoublepage

%% Vytváříme PDF/A-2u
\usepackage[a-2u]{pdfx}

%% Přepneme na českou sazbu a fonty Latin Modern
\usepackage[czech]{babel}
\usepackage{lmodern}
\usepackage[T1]{fontenc}
\usepackage{textcomp}

%% Použité kódování znaků: obvykle latin2, cp1250 nebo utf8:
\usepackage[utf8]{inputenc}

%%% Další užitečné balíčky (jsou součástí běžných distribucí LaTeXu)
\usepackage{amsmath}        % rozšíření pro sazbu matematiky
\usepackage{amsfonts}       % matematické fonty
\usepackage{amsthm}         % sazba vět, definic apod.
\usepackage{bbding}         % balíček s nejrůznějšími symboly
			    % (čtverečky, hvězdičky, tužtičky, nůžtičky, ...)
\usepackage{bm}             % tučné symboly (příkaz \bm)
\usepackage{graphicx}       % vkládání obrázků
\usepackage{fancyvrb}       % vylepšené prostředí pro strojové písmo
\usepackage{indentfirst}    % zavede odsazení 1. odstavce kapitoly
\usepackage{natbib}         % zajištuje možnost odkazovat na literaturu
			    % stylem AUTOR (ROK), resp. AUTOR [ČÍSLO]
\usepackage[nottoc]{tocbibind} % zajistí přidání seznamu literatury,
                            % obrázků a tabulek do obsahu
\usepackage{icomma}         % inteligetní čárka v matematickém módu
\usepackage{dcolumn}        % lepší zarovnání sloupců v tabulkách
\usepackage{booktabs}       % lepší vodorovné linky v tabulkách
\usepackage{paralist}       % lepší enumerate a itemize
\usepackage{xcolor}         % barevná sazba

%%% Údaje o práci

% Název práce v jazyce práce (přesně podle zadání)
\def\NazevPrace{Webová aplikace pro vyhledávání receptů}

% Název práce v angličtině
\def\NazevPraceEN{Web application for searching recipes}

% Jméno autora
\def\AutorPrace{Kristýna Lhoťanová}

% Rok odevzdání
\def\RokOdevzdani{2022}

% Název katedry nebo ústavu, kde byla práce oficiálně zadána
% (dle Organizační struktury MFF UK, případně plný název pracoviště mimo MFF)
\def\Katedra{Katedra softwarového inženýrství}
\def\KatedraEN{Department of Software Engineering}

% Jedná se o katedru (department) nebo o ústav (institute)?
\def\TypPracoviste{Katedra}
\def\TypPracovisteEN{Department}

% Vedoucí práce: Jméno a příjmení s~tituly
\def\Vedouci{doc. Mgr. Martin Nečaský, Ph.D.}

% Pracoviště vedoucího (opět dle Organizační struktury MFF)
\def\KatedraVedouciho{Katedra softwarového inženýrství}
\def\KatedraVedoucihoEN{Department of Software Engineering}

% Studijní program a obor
\def\StudijniProgram{Informatika}
\def\StudijniObor{Programování a vývoj software}

% Nepovinné poděkování (vedoucímu práce, konzultantovi, tomu, kdo
% zapůjčil software, literaturu apod.)
\def\Podekovani{%
Děkuji svému vedoucímu doc. Mgr. Martinovi Nečaskému, Ph.D. za vedení této práce, zejména za cenné rady v oblasti návrhu architektury a přístupu k implementaci, které uplatním i v nadcházejících projektech.
}

% Abstrakt (doporučený rozsah cca 80-200 slov; nejedná se o zadání práce)
\def\Abstrakt{%
Cílem této práce je vyvinout webovou aplikaci pro vyhledávání receptů založenou na agregaci datových sad z~existujících webových stránek s~recepty a~jejich obohacení o~data ze znalostních grafů. Znalostní grafy byly zastoupeny projekty DBpedia a Wikidata, z~nichž byla získána rozšiřující data o~ingrediencích. Data byla extrahována s~pomocí knihovny Apify pro web scraping a~v~dokumentovém modelu uložena do databázového systému Apache CouchDB. Aplikace uživateli poskytuje různé možnosti filtrování výsledků včetně fasetového vyhledávání, k~čemuž využívá platformu Apache Solr. Zaměřuje se zejména na vyhledávání dle ingrediencí. Jedná se o~tzv.~single-page aplikaci implementovanou pomocí knihovny React.js pro uživatelské rozhraní a~frameworku Express.js na straně serveru. Obě části aplikace jsou psány staticky typovaným jazykem TypeScript a~komunikují spolu prostřednictvím REST~API.
}
\def\AbstraktEN{%
This thesis aims to develop a web application for searching recipes. The search for recipes is based on aggregating datasets from the existing recipe websites and extending the data using knowledge graphs. Knowledge graphs were represented by DBpedia and Wikidata projects. These were used to gather data about ingredients. Data were extracted using the Apify web scraping library and stored in the database system Apache CouchDB using the document model. The application provides the user with different options for filtering results, including faceted search. Faceted search is implemented using the Apache Solr platform. The focus is on searching based on ingredients. The web application is a single-page application implemented using the React.js library at the frontend and the Express.js framework at the backend. Both parts of the application are written in statically typed language TypeScript and exchange information through REST API.
}

% 3 až 5 klíčových slov (doporučeno), každé uzavřeno ve složených závorkách
\def\KlicovaSlova{%
{webová aplikace}, {recept}, {znalostní graf}, {propojená data}
}
\def\KlicovaSlovaEN{%
{web app}, {recipe}, {knowledge graph}, {linked data}
}

%% Balíček hyperref, kterým jdou vyrábět klikací odkazy v PDF,
%% ale hlavně ho používáme k uložení metadat do PDF (včetně obsahu).
%% Většinu nastavítek přednastaví balíček pdfx.
\hypersetup{unicode}
\hypersetup{breaklinks=true}

%% Definice různých užitečných maker (viz popis uvnitř souboru)
\include{makra}

%% Titulní strana a různé povinné informační strany
\begin{document}
\include{titulka}

%%% Strana s automaticky generovaným obsahem bakalářské práce

\tableofcontents

%%% Jednotlivé kapitoly práce jsou pro přehlednost uloženy v samostatných souborech
\chapter*{Úvod}
\addcontentsline{toc}{chapter}{Úvod}

Vyhledávání relevantního obsahu je spolu s elektronickou komunikací jednou z~klíčových funkcí internetu. S~rostoucím množstvím dostupných informací se filtrování nalezených výsledků stává stále obtížnějším. Tvůrci webových stránek se často zaměřují spíše na uživatelsky přívětivé interaktivní rozhraní, zatímco optimalizace strojového vyhledávání jde stranou. Pro webové vyhledávače, jmenovitě např. Google, Bing nebo Yahoo, je pak náročné analyzovat obsah těchto stránek po sémantické stránce a~tedy vyhodnotit, zda obsahují užitečné informace k~zodpovězení dotazu uživatele.

V~reakci na tuto problematiku vznikl tzv. \emph{Sémantický Web} neboli Web dat jakožto rozšíření původního Webu dokumentů, tak jak jej známe z platformy \emph{World Wide Web}. Sémantický Web lze vnímat jako globální databázi, nad kterou se lze pomocí speciálního jazyka \emph{SPARQL} dotazovat podobně jako nad tradičními databázovými systémy. Data jsou poskytována v~různých serializacích formátu RDF a~mohou být přímo vnořena do HTML dokumentů nebo zpřístupněna v~samostatných souborech. Tato strukturovaná data nazýváme \emph{propojená} (v~originále \emph{Linked Data}). Umožňují snadnější hledání souvislostí mezi entitami z~různých zdrojů na základě společných slovníků neboli ontologií \citep{semantic-web}.

V posledních letech termín Sémantický Web ustupuje do pozadí a~často je místo něj zmiňován tzv. \emph{znalostní graf} (anglicky \emph{Knowledge Graph}). Začátky fenoménu znalostních grafů bychom mohli datovat do roku 2012, kdy společnost Google představila svůj znalostní graf pro vyhledávání obsahu na webu. K~technologii znalostních grafů se brzy poté přihlásily další velké společnosti včetně firem Microsoft, IBM, Facebook, LinkedIn, Amazon, eBay, Airbnb nebo Uber. Grafový model totiž oproti tradičnímu relačnímu modelu nabízí flexibilnější správu dat z~oblasti sociálních sítí, dopravních spojení, bibliografických citací a~řady dalších odvětví. Výše zmíněné příklady znalostních grafů všechny spadají do kategorie komerčních znalostních grafů, které jsou určeny pro interní využití v~rámci dané firmy. Protikladem jim jsou otevřené znalostní grafy poskytující data k~volnému využití všem uživatelům internetu. Nejvýznamnějšími představiteli otevřených znalostních grafů jsou aktuálně DBpedia, Wikidata, Freebase a YAGO \citep{kg-book}. První dva zmíněné projekty si představíme v~této práci a~integrujeme je s~aplikací na vyhledávání receptů.

Oblast gastronomie je rozvěž vhodným kandidátem k~zapojení do sítě znalostních grafů a~propojených dat. Pro tvůrce webových aplikací je poměrně jednoduché publikovat obsah svých stránek ve formátu strukturovaných dat. Vhodným způsobem je např. vložení RDF reprezentace daných entit (receptů, uživatelů, recenzí) ve formátu JSON-LD\footnote{Koncovka \emph{LD} v~názvu JSON-LD odkazuje na pojem Linked Data.} přímo do hlavičky jednotlivých HTML dokumentů. V~takovém případě je žádoucí použít existující ontologie raději než definovat vlastní, byť by mohly být lépe strukturované a uzpůsobené dané doméně. Využití standardizovaných slovníků usnadňuje webovým vyhledávačům interpretaci stránky a je větší šance, že se aplikace dostane na vyšší příčky vyhledávaných výsledků.

Cílem této bakalářské práce je prozkoumat možnosti využití otevřených dat v~doméně receptů, propojit je s~daty publikovanými na různých webových stránkách shromažďujících recepty a~prezentovat tyto výsledky uživateli ve formě vlastní webové aplikace. Zároveň v~rámci této aplikace poskytnout užitečné možnosti filtrování agregovaných výsledků včetně fasetového vyhledávání. Proces sbě\-ru, konverze a~uložení dat by měl být co nejvíce automatizovaný a~snadno zreprodukovatelný. Práce se nevěnuje přidávání nových receptů prostřednictvím uživatelského rozhraní. Existujících webové stránky totiž obsahují velké množství dat, které lze díky bohaté historii v~podobě hodnocení a~recenzí lépe filtrovat. Navíc by bylo potřeba se vypořádat s~automatickou kalkulací nutričních hodnot receptu z~obsažených surovin, přičemž ne všechny ingredience dokážeme automaticky identifikovat a~získat jejich nutriční hodnoty. V~budoucnu by funkce nahrávání nových receptů měla být přidána spolu s~více lokalizacemi aplikace, registrací uživatelů a~celkovou personalizací obsahu pro přihlášené uživatele. Dále se práce v této fázi nezabývá nasazením, neboť by vyžadovalo větší časové i finanční prostředky na získání dostatečně velkého množství dat a také poměrně robustní databázi pro uložení extrahovaných dat. 

\section*{Volba tématu}
\setcounter{tocdepth}{1}

Příprava jídla je tématem každodenního života a~na webových stránkách, které se této oblasti věnují, má velmi silnou komunitu. Většina z~nás se chystání domácích pokrmů z~ekonomických důvodů nevyhne, takže se hodí mít po ruce sadu receptů pro inspiraci. Typicky máme na recepty různé požadavky - někdo preferuje rychlejší postup, jiný se dívá po ceně ingrediencí nebo nutričních hodnotách. Občas dostaneme chuť na recept z řecké nebo italské kuchyně a~jindy zkrátka chceme experimentovat a~najít recept kombinující našich $5$ oblíbených surovin. Některé ingredience z~receptu nám mohou být neznámé, nebo si jen podle názvu nejsme jistí, zda máme na mysli tu správnou. V~takovém případě musíme stránku s~receptem opustit a~dodatečné informace k~ingredienci vyhledat jinde, pokud na ně aplikace přímo neodkazuje. Zde je příležitost zapojit otevřená data a~namapovat názvy ingrediencí na jejich odpovídající entity ve znalostních grafech. Data pak můžeme začlenit do aplikace a~nabídnout uživateli informace nad rámec samotného receptu, např. popisy a glykemické hodnoty surovin, ilustrační obrázky a podobně. Také můžeme identifikovat ingredience a~tranzitivně recepty ze stejných kategorií. Oproti původní datové sadě tak vytvoříme nové vazby a~poskytneme uživateli rozmanitější filtrování výsledků.

Doména receptů navíc poskytuje spoustu prostoru pro zajímavá rozšíření se zapojením moderních technologií. Uplatnění by zde našlo například počítačové vidění s~rozpoznáváním obrázků. S~dostatečně velkou databází bychom díky němu mohli analyzovat fotografii hotového pokrmu a~nalézt příslušný recept. Usnadnili bychom tak uživateli práci v~situacích jako je návštěva restaurace, při které návštěvníkovi zachutnalo servírované jídlo a~chtěl by si jej později připravit v domácích podmínkách. Dalším uplatněním strojového učení by mohlo být vyhledávání na základě příkazů v~přirozeném jazyce. Namísto zdlouhavého zadávání nejrůznějších filtrů by stačilo aplikaci položit dotaz: \uv{Jaké recepty z~italské kuchyně mohu vyrobit z~kuřete, rajčat a~parmazánu?}. Této problematice se věnuje například projekt FoodKG konstruující nad recepty a~ingrediencemi znalostní graf \citep{food-kg}. Ruku v~ruce s touto funkcionalitou jde hlasové zadávání, které by se hodilo zapojit nejen ve fázi vyhledávání receptů, ale také například pro hands-free ovládání aplikace. Uživatel by měl možnost diktovat příkazy k~přečtení části receptu, pokud zrovna pracuje na jeho přípravě a~nemá volné ruce k~listování obsahem. Využití by našlo i~populární \emph{full-text} vyhledávání, pomocí kterého lze snadno objevit recepty na základě klíčových slov v~popisku receptu, postupu či recenzích. V~komerční sféře by se nabízelo propojení s~online supermarkety, konkrétně zrychlení nákupu pomocí vyhledávání surovin k~vybranému receptu. S~tímto konceptem již na svých stránkách pracuje firma rohlik.cz, nabídka receptů a~možnosti filtrování jsou ale omezené. Nepochybně by se hodilo integrovat také doporučovací systém pro ještě snadnější nalezení relevantních výsledků. Aplikace má velký prostor pro škálování objemu dat, přičemž datasety mohou být následně použity jako podklad pro strojové učení.
%%% Fiktivní kapitola s ukázkami sazby

\chapter{Analýza}

V~této kapitole si zadefinujeme požadavky na funkcionalitu naší aplikace. Také se v kontextu požadavků podíváme na existující webové stránky s~recepty a~provedeme diskuzi nad jejich funkcemi, možnými vylepšeními a~rozšířeními. Následně si rozebereme různé alternativy dostupných datových sad a srovnáme jejich výhody i~nevýhody vzhledem k~požadavkům aplikace.

\section{Požadavky aplikace}

Nyní si rozebereme požadavky na naši aplikaci, které můžeme rozdělit do skupin funkčních a nefunkčních požadavků. Funkční požadavky popisují konkrétní funkcionalitu systému, zabývají se vstupem od uživatele a prezentací výstupu. Díky tomu je lze poměrně snadno definovat a testovat jejich naplnění ve funkční aplikaci. Nefunkční požadavky se naopak na konkrétní vstup nevážou a místo toho popisují vlastnosti a omezení, které by měl systém splňovat. Zjednodušeně lze říci, že funkční požadavky popisují, co má systém dělat, zatímco nefunkční požadavky specifikují, jaký má systém být \citep{app-requirements}.

\subsection{Funkční požadavky}

Následuje výčet funkcionalit, které by aplikace svým uživatelům měla nabídnout. Uživatelé mohou mít různé role od běžného návštěvníka stránky po administrátora nebo vývojáře integrujícího data do jiného systému.

\subsubsection{Běžný uživatel}

\begin{enumerate}
    \item Aplikace poskytuje uživatelské rozhraní pro vyhledávání receptů na základě ingrediencí, klíčových slov, času přípravy, hodnocení a nutričních hodnot.
    \item Aplikace umožňuje kombinovat libovolné množství vyhledávácích filtrů.
    \item Aplikace podporuje zadávání vlastních i předdefinovaných ingrediencí prostřednictvím našeptávače.
    \item Aplikace podporuje fasetové vyhledávání, tedy u nabízených možností zobrazuje počet receptů, které se po zvolení daného filtru zobrazí.
    \item Aplikace poskytuje možnost smazání všech vyhledávacích filtrů jedním kliknutím, ale také mazání po jednom filtru.
    \item Aplikace zobrazuje uživateli všechny nalezené výsledky bez omezení na maximální počet výsledků.
    \item Aplikace při otevření vyhledávací obrazovky bez zadaných filtrů zobrazuje všechny recepty, které má v databázi.
    \item Aplikace umožňuje zobrazení detailu receptu rozkliknutím nalezeného výsledku.
    \item Aplikace zobrazuje pouze recepty s titulní fotografií.
    \item Aplikace na vyhledávací stránce pro každý nalezený recept zobrazuje jeho název, popis, obrázek, čas přípravy, hodnocení a počet recenzí.
    \item Aplikace nabízí náhledy všech ingrediencí u vyhledaných receptů a zvýrazňuje aktuálně vyhledávané ingredience.
    \item Aplikace umožňuje listování nalezenými výsledky prostřednictvím systému stránkování, nikoli nekonečným posouváním stránky.
    \item Aplikace plně podporuje navigaci v rámci historie prohlížeče včetně přidávání a odebírání filtrů a listováním více stranami výsledků.
    \item Aplikace na detailní stránce každého receptu zobrazuje název, hodnocení, počet recenzí, popis, čas přípravy, fotografii, ingredience, postup přípravy a nutriční hodnoty.
    \item Aplikace zvýrazňuje ingredience na detailní stránce receptu, ke kterým má dodatečné informace.
    \item Aplikace přesměrovává na obrazovku s detailem ingredience po kliknutí na zvýrazněnou ingredienci.
    \item Aplikace zobrazuje na detailní stránce ingredience následující informace nebo jejich podmnožinu: název, popis, obrázek, nutriční hodnoty, náhrady, kategorie a níže recepty obsahující tuto ingredienci, které lze otevřít stejně jako z vyhledávací obrazovky.
    \item Aplikace má nezávisle na otevřené stránce viditelný ovládací panel s možností navigace na vyhledávací obrazovku.

\end{enumerate}

\subsubsection{Externí systém}

\begin{enumerate}
    \item Aplikace poskytuje REST API endpointy pro získání dat k receptům a ingrediencím.
    \item Aplikace vkládá JSON-LD reprezentaci dat do hlaviček dokumentů s recepty a ingrediencemi.
    \item Aplikace podporuje navigaci a vyhledávání receptů přes url adresy s query parametry.
\end{enumerate}

\subsection{Nefunkční požadavky}

\section{Dostupné datové sady}

V této sekci je vyhrazen prostor pro analýzu různých veřejně dostupných datasetů z domény receptů. Nejedná se ani zdaleka o kompletní výčet, měly by ale být představeny nejznámější alternativy, které by mohly být vybrány jako podklad pro obsah aplikace. 

\subsection{Recipe1M+}

V~první fázi analýzy se zaměříme na veřejně dostupná zdrojová data s~recepty, která by mohla posloužit jako podklad pro naši databázi. Jedním z~nejdůležitějších projektů v~této oblasti je \emph{Recipe1M+}, strukturovaný korpus obsahující přes $1$~milion receptů a $13$ milionů souvisejících obrázků jídla. Aktuálně se jedná o~největší veřejně dostupnou sadu receptů. Dataset je dostupný pouze přihlášeným uživatelům z~ověřené organizace a je povoleno jej využívat výhradně pro účely studia a výzkumu. Z~celkového počtu $1$~milionu receptů obsahuje $50\,000$ receptů s~nutričními informacemi \citep{marin2019learning}. V~naší aplikaci preferujeme nutriční hodnoty zahrnout, pokud jsou dostupné na zdrojové stránce receptu. Měli bychom tedy k~dispozici $50\,000$ dokumentů s~touto informací. Ostatní data jsou určena přednostně pro strojové zpracování prostřednictvím trénování modelů.

Celková velikost datové sady se pohybuje v~řádu stovek gigabytů, samotné JSON dokumenty se strukturovanými recepty z~adresáře \texttt{layers} se ale vejdou do $2~GiB$, tudíž by byly vhodné pro potřeby této práce limitované omezenou výpočetní kapacitou. Lze odtud využít $1\,029\,720$ receptů obsahujících název, url, ingredience a~postup přípravy. Odkazy na ilustrační fotografie jsou u~$402\,760$ z~těchto receptů. Pro příjemnější uživatelský zážitek se omezujeme pouze na recepty s~obrázky, takže jsme z~datasetu Recipe1M+ schopni použít přibližně $400\,000$ receptů, pokud akceptujeme absenci nutričních hodnot. Bylo by spíše obtížnější z~tohoto datasetu identifikovat názvy ingrediencí, neboť jsou suroviny uloženy včetně jejich množství a~jednotek měření v~rozmanitém formátu.

\subsection{Open Recipes}

Dalším významným aktérem na poli volně dostupných receptů je iniciativa \emph{Open Recipes}. Autoři Finkler, Shiflett a Birkebæk projekt představují jako otevřenou databázi záložek s~recepty. Pojem záložky je použit z~důvodu absence instrukcí k~přípravě receptu. Dataset má sloužit pouze k~vyhledání receptu a~pro detailní informace má být uživatel přesměrován na zdroj s~kompletním receptem \citep{open-recipes}. Tohoto přístupu úspěšně využívají některé z~vyhledávačů receptů, např. populární aplikace \emph{SuperCook}. Naše aplikace si ale klade za cíl zpracovat i~stránky s~detaily receptů, ze kterých lze dále pokračovat na detaily ingrediencí s~informacemi ze znalostních grafů. Projekt Open Recipes tedy pro náš scénář nebude vhodnou volbou.

\subsection{Food.com Recipes and Interactions}

Rozsáhlý dataset \emph{Food.com Recipes and Interactions} s~téměř $200\,000$ recepty extrahovanými z~webové stránky \emph{Food.com} (původního GeniusKitchen) je publikován na portálu \emph{Kaggle}, který shromažďuje podklady pro strojové učení. Datová sada pokrývá $18$ let interakce uživatelů včetně hodnocení, počtu recenzí i~konkrétních reakcí \citep{shuyang_li_2019}. Kromě základních informací obsahuje také nutriční hodnoty receptů, datum publikování a~rovněž normalizovaná jména ingrediencí. Ta byla získána parsováním originálního textu surovin, kvůli čemuž nejsou vždy zcela spolehlivě přesná (např. ve jménech často zůstala jednotka měření z~původního textu). Unikátních ingrediencí je k~dispozici kolem $8\,000$, což by měl být dostačující základ pro hledání linků s~entitami otevřených znalostních grafů. Zároveň ve srovnání s~předchozími projekty nabízí nejbohatší informace k jednotlivým receptům.

Nevýhodou datasetu je jeho primární určení pro strojové zpracování. Byl vytvořen jako podklad pro generování personalizovaných receptů na základě dřívějších preferencí uživatele \citep{majumder-etal-2019-generating}. Syrová data nejsou zamýšlena pro přímou prezentaci, což se negativně odráží na jejich přesnosti a estetice. Slova jsou občas zařazena do špatných kategorií a~problematický je zejména plně \emph{lowercase} formát textu, ze kterého nejsme schopni zpětně zrekonstruovat originální text receptu. Dataset bychom tedy nemohli použít samostatně, ale pouze v~kombinaci s~vlastní extrakcí dat, která by respektovala velikost písma a~lépe se vypořádala s~parsováním jednotlivých kategorií.

Tento problém je poměrně snadno řešitelný díky struktuře stránky Food.com. Z~id receptu lze jednoduše složit url ve formátu \texttt{www.food.com/recipe/id} a~navíc aplikace podporuje koncept propojených dat, tedy poskytuje recepty ve strukturovaném RDF formátu. Do HTML hlaviček všech dokumentů s~recepty vkládá JSON-LD serializaci dle ontologie \emph{Schema.org}. Z~připraveného datasetu bychom tedy mohli využít identifikátory receptů a~normalizované ingredience, pro každý recept extrahovat jeho JSON-LD a spojit informace dohromady. Zároveň bychom si ušetřili práci s~převáděním receptů do JSON-LD formátu a připravené soubory rovnou vložili do hlaviček dokumentů. Nevytvářeli bychom nové entity receptů, pouze bychom změnili prezentační vrstvu RDF dat. Identifikátory entit v podobě IRI by tedy zůstaly nezměněné.

\subsection{FoodKG}

Přímo v~oblasti znalostních grafů figuruje projekt \emph{FoodKG}, který je postaven nad sadou receptů z~již zmíněného datasetu Recipe1M+. Recepty doplňuje o~podrobnější data k~ingrediencím ze stránky The~Cook's~Thesaurus a~definuje vlastní ontologii. Model ontologie je navržen pro zodpovídání dotazů na recepty dle ingrediencí s~přihlédnutím k~individuálním potřebám uživatele, jako jsou alergie a~intolerance na určité složky potravin.

Vývojáři projektu FoodKG zpřístupňují skripty k~extrakci dat z~encyklopedie The~Cook's~Thesaurus a~k~vytvoření znalostního grafu. Neposkytují ale žádné nové recepty nad rámec datové sady Recipe1M+, naší horní hranicí by tedy bylo $50\,000$ receptů s~nutričními hodnotami (viz sekce \emph{Recipe1M+}). Ontologie publikovaná na webových stránkách projektu obsahuje $75$ entit ingrediencí, které kromě obecného popisu poskytují informace o~glykemickém indexu, obsahu lepku a~možných náhradách dané ingredience. Výhodou je připravený RDF formát, nad kterým se lze snadno dotazovat pomocí jazyka SPARQL. Autoři Chen~a~kol. uvádějí ukázky dotazů, vyberme například dotaz vracející recepty, které obsahují banán a~zároveň neobsahují vlašské ořechy \citep{food-kg}:

\begin{code}
@PREFIX food: <http://purl.org/heals/food/>
@PREFIX ingredient: <http://purl.org/heals/ingredient/>
SELECT DISTINCT ?recipe
WHERE {
    ?recipe food:hasIngredient ingredient:Banana .
    FILTER NOT EXISTS {
        ?recipe food:hasIngredient ingredient:Walnut .
    }
}
\end{code}
%$

\subsection{Generování vlastního datasetu}

Pokud se nespokojíme s~žádnou z~dostupných datových sad, případně potřebujeme data rozšířit a~posbírat je přímo ze zdroje, využijeme metodu zvanou \emph{web scraping}. V~rámci tohoto procesu musíme analyzovat cílovou stránku z~pohledu získávání a~prezentace dat. S~využitím vývojářským nástrojů ve webovém prohlížeči můžeme přes panel \texttt{Network} sledovat požadavky, které aplikace odesílá na svůj server a~v~mnoha případech se na toto interní API dokážeme napojit a~získat data ve strukturované podobě. Aplikace typicky pracují s~REST~API, GraphQL API nebo jejich kombinací a~standardně data poskytují ve formátu JSON. Pokud žádný fetch request pro získávání potřebných dat neobjevíme, musíme informace extrahovat přímo z~HTML dokumentu prostřednictvím CSS selektorů. V~obou případech budeme aplikaci posílat GET requesty, ať už na její backend pro strukturovaná data nebo na frontend pro HTML dokumenty k následnému parsování.

Problematická je kategorie aplikací, které data nezískávají s~využitím transparentních fetch requestů a~zároveň potřebují spouštět JavaScript kód pro vygenerování obsahu. Zde nestačí pouhé poslání GET requestu přes HTTP, neboť odpověď neobsahuje žádná relevantní data uvnitř HTML. Pro zvládnutí tohoto typu stránek potřebujeme zapojit automatizaci webového prohlížeče. Nejznámějšími projekty, které se této automatizaci věnují, jsou Selenium\footnote{https://github.com/SeleniumHQ/selenium}, Puppeteer\footnote{https://github.com/puppeteer/puppeteer}, Playwright\footnote{https://github.com/microsoft/playwright} a~Cypress\footnote{https://github.com/cypress-io/cypress} pro automatizaci testování \citep{selenium-ecosystem}. Všechny ze zmíněných projektů jsou open-source.

Během posílání requestů můžeme rovněž narazit na různé formy blokování, od limitu maximálního počtu requestů z~jedné IP adresy přes povinné autorizační tokeny až po captcha testy řešitelné pouze s~využitím umělé inteligence. Některé aplikace navíc kontrolují tzv. otisk webového prohlížeče. Jedná se o~sadu informací k~zařízení uživatele, jmenovitě data o~konkrétním hardwaru, operačním systému a~webovém prohlížeči včetně konfigurace \citep{browser-fingerprints}. Také se při neopatrnosti může stát, že server aplikace zahltíme příliš velkým množstvím paralelních requestů, čímž prodloužíme dobu odezvy nebo zpracování dalších requestů dočasně zcela znemožníme.

Stejně jako v~jiných oblastech se hodí využít nástroj, který co nejvíce běžných problémů vyřeší za nás. Na poli open-source nástrojů pro extrakci dat si vedoucí pozici drží knihovna Scrapy\footnote{https://github.com/scrapy/scrapy} psaná v~jazyce Python, která nabízí celou řadu pokročilých funkcí proti blokování requestů. Pro potřeby této práce by ale vzhledem k~rozsáhlejší osobní zkušenosti byla vhodnou volbou knihovna Apify\footnote{https://github.com/apify/apify-js} pro Node.js. V arzenálu má zpracování HTTP requestů s~následným parsováním HTML pomocí knihovny Cheerio\footnote{https://github.com/cheeriojs/cheerio}, ale také automatizaci webového prohlížeče s~využitím knihoven Puppeteer nebo Playwright, včetně generování otisků webového prohlížeče. Navíc zajišťuje rotaci IP adres, čímž snižuje množství zablokovaných requestů. IP adresy lze v rámci placeného účtu získat přímo od firmy Apify, nebo na vstupu poskytnout seznam vlastních. Obecně preferujeme program nespouštět z~osobní IP adresy, neboť riskujeme, že nás stránka někdy i~natrvalo zablokuje, případně se naše IP adresa dostane na veřejný seznam adres doporučených k~blokování.

S~dostatkem času, výpočetních prostředků, IP adres pro rotování a~s~velikou kapacitou úložiště bychom byli schopni zpracovat většinu vybraných aplikací s recepty. Pro každou stránku bychom napsali dedikovaný program a postupně extrahovali data z~celé stránky. Recepty z~různých aplikací bychom uložili ve sjednoceném formátu a~výsledkem by byl kvalitní dataset s~maximálním množstvím dat, které lze od zdrojových stránek získat. Práce ovšem necílí na datovou sadu takovéto velikosti. Místo toho se zaměřuje na vytvoření infrastruktury nad podmnožinou receptů, kterou bude možné libovolně škálovat dle možností dalšího vývoje. V~případě vlastní extrakce dat bychom si tedy vybrali dva až tři zástupce aplikací, navrhli pro ně jednoduché řešení extrakce dat a~omezili počet sesbíraných výsledků na rozumnou hodnotu. Vhodným kandidátem by jednoznačně byla zmíněná stránka Food.com, která obsahuje přes $500\,000$ receptů a~pro cca $200\,000$ z nich máme k dispozici unikátní identifikátory skrze dataset z~platformy Kaggle. Navíc dokumenty s~recepty obsahují JSON-LD reprezentaci v hlavičce HTML. Pro každý recept se známým id by tedy stačilo vytvořit url, poslat na něj GET request a~z~HTML odpovědi extrahovat JSON-LD data. Podobně bychom mohli zpracovat recepty ze stránky Allrecipes, kde jsou v~detailech receptů rovněž publikována JSON-LD data. Url receptů by mohl objevit přímo náš program během procházení stránky nebo bychom mohli využít url adresy z datasetu Recipe1M+.

\subsection{Znalostní grafy}


%%% Fiktivní kapitola s ukázkami citací

\chapter{Návrh}

V~této kapitole si představíme vzhled uživatelských obrazovek, ze kterého vyjdeme při tvorbě uživatelského rozhraní založeného na komponentách knihovny Material UI. Také se budeme věnovat návrhu architektury od extrakce a~uložení dat přes přípravu vyhledávacích indexů až po prezentaci těchto dat v~rámci funkční aplikace.

\section{Návrh uživatelského rozhraní}

Přestože vyvíjíme single-page aplikaci, počítáme s~více obrazovkami pro pohodlnější navigaci. S~využitím knihovny React Router DOM\footnote{https://www.npmjs.com/package/react-router-dom} dokážeme simulovat existenci libovolného množství obrazovek a~zároveň zůstat na jedné stránce bez potřeby opětovného načítání. Tím se odlišíme od tradičních statických aplikací, kterým při každé změně URL včetně query parametrů musí server v~odpovědi poslat odpovídající HTML obsah. Náš přístup má ovšem nevýhodu z~pohledu strojového zpracování, neboť pro vygenerování obsahu stránky potřebujeme v~prohlížeči spustit JavaScript kód. Tím znemožníme zpracování naší aplikace prostřednictvím pouhých HTTP požadavků, což je podstatně jednodušší a~ekonomičtější varianta ve srovnání s~automatizací celého webového prohlížeče. Tento nedostatek ale kompenzujeme transparentním REST API, přes které si lze vyžádat strukturovaná data pomocí HTTP požadavků. Zároveň usnadníme strojové zpracování vyhledávačům, které automatizaci prohlížeče využívají, neboť v~detailech receptů a~ingrediencí zahrneme jejich JSON-LD reprezentaci.

Aplikaci složíme ze $3$ základních uživatelských obrazovek: vyhledávání receptů, detail receptu a~detail ingredience. Všechny obrazovky musí být responzivní a~poradit si s~proměnlivou velikostí obrazovky.

\subsection{Vyhledávání receptů}

Domovskou stránku bude tvořit vyhledávání receptů na základě různých kritérií. Primárně bude k~dispozici výběr požadovaných ingrediencí, sekundárně filtry klíčových slov, kategorií, času přípravy, hodnocení a~nutričních hodnot. Všechny filtry bude možné odstranit samostatně i~najednou pomocí společného tlačítka pro smazání. Pro získání přesnějších výsledků bude při vyplňování filtrů k~dispozici našeptávač, který zobrazí známé možnosti a~spolu s~nimi počty receptů, které jsou při výběru tohoto nastavení k~dispozici. Výsledky budou zobrazovány na stránkách s~$24$ nebo $30$ kartami receptů. Přepínání stránek bude umístěno standardně ve spodní části stránky a~zároveň bude aktuální stránka figurovat v~query parametrech pro přímočarou podporu navigace v~historii prohlížeče. Karta receptu bude obsahovat název, popis, obrázek, hodnocení, počet recenzí, čas přípravy a~počet instrukcí. Navíc bude možné rozbalit seznam ingrediencí, ve kterém budou zvýrazněny aktuálně vyhledávané ingredience. Uživatel bude přesměrován na obrazovku s~detailem receptu při stisknutí tlačítka \texttt{View} nebo při kliknutí na obrázek receptu. Konkrétní rozložení obrazovek viz obrázek \ref{obr02:desktop-search-view} pro desktopová zařízení a \ref{obr02:mobile-search-view} pro mobilní zařízení.

\begin{figure}[p]\centering
\includegraphics[width=140mm]{../img/desktop-search-view}
\caption{Obrazovka vyhledávání receptů pro desktopová zařízení.}
\label{obr02:desktop-search-view}
\end{figure}

\begin{figure}[p]\centering
\includegraphics[height=230mm]{../img/mobile-search-view}
\caption{Obrazovka vyhledávání receptů pro mobilní zařízení.}
\label{obr02:mobile-search-view}
\end{figure}

\begin{figure}[p]\centering
\includegraphics[width=140mm]{../img/detail-view}
\caption{Obrazovka detailu receptu.}
\label{obr02:detail-view}
\end{figure}

\subsection{Detail receptu}

Na obrazovce s~konkrétním receptem bude obsažen název, popis, obrázek, autor, datum publikování, hodnocení s~počtem recenzí, nutriční hodnoty a~samozřejmě ingredience a~postup přípravy. Rozložení pro větší obrazovky viz obrázek \ref{obr02:detail-view}, pro menší obrazovky se všechny karty zobrazí v $1$~sloupci analogicky k~rozložení \ref{obr02:mobile-search-view}. Jako budoucí rozšíření by bylo možné implementovat modul recenzí. V~první fázi by recenze byly pouze extrahovány ze zdrojových datasetů, v~další fázi by uživatelé mohli nové recenze přidávat prostřednictvím naší aplikace.

\subsection{Detail ingredience}

Na obrazovce detailu ingredience budou prezentována data z~otevřených znalostních grafů. Zaměříme se primárně na jméno, popis a~obrázek ingredience, které dle dostupných informací doplníme o~nutriční hodnoty, kategorie, místo původu a~další zajímavosti.

\section{Návrh architektury}

Při návrhu architektury aplikace budeme vycházet z~osvědčené kombinace technologií označované zkráceně jako \emph{MERN}. Pojmenování vzniklo složením prvních písmen klíčových technologií, tedy MongoDB, Express, React a~Node. Pro potřeby naší aplikace se této skupiny nebudeme držet zcela striktně a~systém MongoDB nahradíme rovněž dokumentovým databázovým systémem Apache \,CouchDB. Navíc mezi technologie zařadíme platformu Apache Solr pro pokročilejší vyhledávání receptů. Architektura naší aplikace znázorňující klíčové technologie a jejich propojení je ilustrována schématem \ref{obr02:architecture}. 

\begin{figure}[h!]\centering
\includegraphics[width=140mm]{../img/architecture}
\caption{Schéma architektury, adaptováno z diagramu MERN aplikace \citep{mern-stack}.}
\label{obr02:architecture}
\end{figure}

Předzpracování dat bude klíčovou fází řešení spolu s~prezentační vrstvou na straně klienta. Serverová část aplikace bude vystupovat pouze jako prostředník mezi klientem a~databází, respektive platformou Solr pro vyhledávací dotazy. Jejím úkolem bude zprostředkování požadovaných dat z~perzistovaného úložiště, která téměř beze změny předá klientovi. Tento přístup si vybere daň v~podobě obsáhlejší databáze, neboť budeme ukládat i~taková data, která bychom dokázali vygenerovat z~ostatních informací. Nejdůležitější instancí tohoto opakování dat bude uložení JSON-LD reprezentace a~zároveň strukturovaných dat v~rámci každého dokumentu receptu. Strukturovaná data by bylo možné při každém dotazu odvodit z JSON-LD nebo naopak. Přesunuli bychom ale komplexitu převodu z~jednorázové fáze předzpracování na samotnou aplikaci, ať už na straně serveru či klienta. Příprava dat pro prezentaci by navíc při každém dotazu trvala o~něco déle, což by se při větším množství dat mohlo negativně odrazit na svižnosti aplikace a~tranzitivně na uživatelském zážitku.

\subsection{Příprava dat}

V~předchozí kapitole jsme si představili řadu alternativ pro získání datasetů s~recepty a~následně doplňujících informací k~ingrediencím. Dle požadavků aplikace potřebujeme minimálně $50\,000$ receptů z~aspoň $2$ různých zdrojů. Také vyžadujeme integraci $2$ nebo více znalostních grafů s~otevřenými daty. Nemáme spodní limit na počet ingrediencí, které musíme ze znalostních grafů extrahovat. Záleží totiž na úspěšnosti propojení našich ingrediencí s~entitami ze znalostních grafů, která se projeví až při praktickém testu.

Přípravu dat můžeme dále rozdělit na $2$ základní fáze a~to extrakci a čištění dat. Ne všechna data jsou totiž vhodná pro přímou prezentaci uživateli. Jak jsme zmiňovali v~minulé kapitole, volně dostupné datasety s~recepty často cílí spíše na oblast strojového učení. Extrahovaná data je tedy potřeba manuálně zkontrolovat a navrhnout heuristiku, pomocí které bude většina dat normalizována, případně odstraněna při nesplnění zadaných kritérií. Vzhledem k charakteru problému a~omezené časové dotaci nelze na každou datovou sadu aplikovat deterministické řešení, které by eliminovalo všechny anomálie, proto se v~některých případech musíme spokojit s~heuristikou.

\subsubsection{Extrakce dat}

Na tomto místě je vhodné rozhodnout, které ze zdrojů dat popsaných v~předchozí kapitole nakonec použijeme v~rámci našeho řešení. U~dat k~ingrediencím máme na výběr znalostní grafy DBpedia a~Wikidata, případně méně obsáhlý RDF dataset z~projektu FoodKG. Pro maximalizaci počtu nalezených výsledků se zaměříme na grafy DBpedia a~Wikidata. V~kategorii receptů zvolíme kombinaci statických datových sad s~recepty a~generování vlastních datasetů pomocí procesu web scraping. Během implementace vlastní extrakce dat zapojíme knihovnu Apify pro Node.js a~její koncept tzv.~\emph{actorů}, což jsou programy určené primárně pro cloudovou platformu Apify, kde jsou spouštěny uvnitř Docker kontejnerů. Mohou mít za úkol automatizaci libovolných úkonů prováděných ve webovém prohlížeči, od jednoduchého posílání e-mailů až po extrakci dat z~komplexních webových stránek. Actory lze pomocí nástroje Apify CLI\footnote{https://www.npmjs.com/package/apify-cli} spouštět i~lokálně, čehož pro jednodušší konfiguraci využijeme v~našem řešení.

\paragraph{Food.com}\mbox{}\\

Vzhledem k~požadavkům definovaným v~předchozí kapitole nám bude vyhovovat dataset Food.com Recipes and~Interactions dostupný na platformě Kaggle, z~něhož jsme schopni získat přibližně $180\,000$ identifikátorů receptů a~také seznam normalizovaných ingrediencí. Dle provedené analýzy není vhodné použít textová data v~prezentační vrstvě vzhledem k jejich lowercase formátu. Navrhneme tedy řešení z~oblasti web scrapingu, které na vstupu přijme URL adresy s~detaily receptů, na každou adresu odešle požadavek GET a~z~HTML odpovědi extrahuje JSON-LD data. Program bude mít možnost získat přes CSS selektory libovolná data z~načteného HTML, pokud by v JSON-LD reprezentaci nebyla obsažena, nebo byla méně strukturována. Programu tedy přidělíme také zodpovědnost za tvorbu strukturovaných dat, která do vygenerovaného datasetu uloží ke každému receptu spolu s~jeho JSON-LD podobou. Strukturovanými daty zde rozumíme čas přípravy, počet porcí, klíčová slova, která jsou v~JSON-LD uložena ve společném řetězci namísto pole řetězců, hodnocení receptu s~počtem recenzí, nutriční hodnoty s~jednotkami měření a~ingredience s~množstvím a~jednotkou oddělenými od ostatního textu.

Extrahované výsledky uložíme do společného JSON souboru, který následně sloučíme s~informacemi z~datasetu Food.com Recipes and~Interactions. Poskytnutý JSON-LD například neobsahuje kompletní informace o~autorovi, ale pouze jeho jméno. Dle samotného jména nejsme schopni autora jednoznačně identifikovat a~zjistit odkaz na jeho profil v~rámci aplikace Food.com. URL adresa autora je totiž sestavena z~jeho unikátního id, které máme k~dispozici právě v~datasetu z~Kaggle. Dále budeme chtít extrahované recepty rozšířit o~normalizované ingredience, abychom nemuseli navrhovat vlastní heuristiku a~usnadnili si pozdější mapování ingrediencí na entity ze znalostních grafů. Po sloučení všech potřebných dat provedeme finální čištění a následně recepty jako JSON dokumenty uložíme do databáze.

Výše popsané řešení extrakce dat z~Food.com má nevýhodu z~pohledu škálovatelnosti. Maximální počet receptů, které jsme schopni získat, je roven počtu receptů v~datasetu z~Kaggle. Celkový počet receptů na stránce Food.com se od doby pořízení datasetu zvětšil více než dvakrát na aktuálních $526\,851$ receptů. Nicméně i~s~naším zjednodušeným programem vyžadujícím připravené URL adresy detailů receptů jsme schopni získat téměř kompletní data. Zmíněný dataset Recipe1M+ v~době psaní této práce obsahuje téměř $510\,000$ URL adres receptů z~aplikace Food.com. Při potřebě většího škálování bychom mohli využít tato URL, neměli bychom k~nim ovšem normalizované ingredience a~byli bychom omezeni striktně akademickým využitím. Pro účely naší práce se spokojíme s~horní hranicí $180\,000$ receptů s~normalizovanými ingrediencemi. Tyto recepty jsou dle autorů datasetu Majumdera a~kol. podmnožinou receptů z~let $2000$-$2018$, které mají aspoň $3$~kroky postupu a~počet ingrediencí v rozmezí $4$ a $20$ \citep{majumder-etal-2019-generating}. Kód souvisejícího projektu pro generování personalizovaných receptů je dostupný jako open-source na platformě GitHub, lze tedy předpokládat, že datovou sadu lze využívat bez omezení.

\paragraph{Allrecipes}\mbox{}\\

Jako další zdroj receptů si vybereme webovou aplikaci Allrecipes. Pro ni sice nemáme k~dispozici podrobný dataset jako u~stránky Food.com, vystačíme si ale s~vlastní extrakcí dat prostřednictvím Apify actora. Mohli bychom využít prakticky stejnou šablonu, jako u~programu pro zpracování Food.com. S~využitím datasetu Recipe1M+ dokážeme získat $49\,000$ URL adres detailů receptů. Poměrně snadno bychom ale dokázali navrhnout pokročilejší řešení extrakce dat, které by dynamicky procházelo celou webovou stránku Allrecipes, našlo detaily všech receptů a~z~nich extrahovalo aktuální data. Tímto přístupem bychom odstranili závislost na datové sadě Recipe1M+ a~získali větší počet výsledků, ale pro nalezení všech receptů bychom museli zpracovat více požadavků. Celkový počet receptů na Allrecipes se aktuálně pohybuje kolem $50\,000$, což lze zjistit spuštěním vyhledávání bez jakýchkoli nastavených filtrů. Jednou z možností by bylo projít všechny kategorie, jinou zase využití interního API z vyhledávací stránky, přes které lze získat URL adresy $24$ receptů v~rámci $1$~požadavku.

Webová aplikace Allrecipes nabízí svým uživatelům ve vztahu k~ingrediencím vyhledávání dle surovin a~také možnost přizpůsobit množství ingrediencí požadovanému počtu porcí. Tato skutečnost naznačuje, že si aplikace přísady interně spravuje ve strukturované podobě, přestože v~přiloženém JSON-LD je poskytuje jako prostý text včetně množství a jednotky měření. Zaměřme se na konkrétní ingredienci uvnitř HTML dokumentu vybraného receptu. Můžeme si povšimnout, že v~atributech příslušného \texttt{input} elementu jsou uložena strukturovaná data ingredience v~následujícím formátu (ukázka z~receptu \texttt{92462} pro surovinu kuřecí vývar):

\begin{code}
<input
    class="checkbox-list-input"
    data-tracking-label="ingredient clicked"
    data-quantity="½"
    data-init-quantity="0.5"
    data-unit="cup"
    data-ingredient="chicken broth"
    data-unit_family="volumetric"
    data-store_location="Soup"
    type="checkbox"
    value="(14.5 ounce) can chicken broth"
    id="recipe-ingredients-label-92462-0-4">
\end{code}
%$

Z atributů elementu \texttt{input} dokážeme rozpoznat název, množství i~jednotku ingredience. Díky tomu získáme ještě přesnější data, než prostřednictvím normalizovaných ingrediencí z~datasetu Food.com Recipes and~Interactions.

\paragraph{Znalostní grafy}\mbox{}\\

V~první fázi extrakce dat ze znalostních grafů, ať už DBpedia či Wikidata, potřebujeme identifikovat entity ingrediencí, které dokážeme namapovat na jména surovin z~jednotlivých receptů. Pro každou entitu ingredience vyjádřenou pomocí IRI pak extrahujeme vybrané informace. U grafu DBpedia budeme cílit především na:
\begin{itemize}
    \item jméno
    \item popis
    \item obrázek
    \item kategorie
    \item místo původu
\end{itemize}

Z~nutričních hodnot se zaměříme na energii v~kaloriích nebo kilojoulech, dále na obsah tuku, sacharidů, bílkovin, vlákniny, cholesterolu a~cukru. Aktuálně se zabýváme pouze anglickou lokalizací aplikace, všechna textová data tedy omezíme na anglické výsledky. Jedinou povinnou informací bude název (label) ingredience, všechna ostatní data budou nepovinná, neboť se formát i~množství dat napříč ingrediencemi výrazně liší.

\subsubsection{Čištění dat}

V~rámci fáze čištění dat potřebujeme extrahovaná data převést do formátu vhodného k~prezentaci koncovému uživateli. Jednotlivé kroky procesu čištění mohou být rozloženy do více míst přípravy dat. Již během extrakce dat probíhá odstranění mezer a~znaků nového řádku na okrajích řetězců. Dále je potřeba se vypořádat se znaky, které jsou kvůli vnoření v~HTML dokumentu kódovány jinými znaky, aby bylo zajištěno jejich korektní zobrazení. Takové znaky se vyskytují např. v~extrahovaných JSON-LD dokumentech, před jejich uložením do databáze tedy provedeme dekódování. Rekurzivně projdeme obsah každého objektu načteného z~JSON-LD dokumentu a~všechny řetězce dekódujeme s~využitím open-source knihoven pro Node.js. V~našem řešení integrujeme knihovny \texttt{html-escaper} a~\texttt{html-entities} dostupné přes správce balíčků npm.

Dále jsme se rozhodli z~vyhledávání vyřadit recepty bez fotografie, které lze identifikovat a~přeskočit již během fáze extrakce nebo následně při ukládání do databáze, případně až při tvorbě dokumentů pro vyhledávací platformu Solr. Zvolíme poslední způsob, recepty tedy uložíme do vlastní databáze bez ohledu na přítomnost jejich obrázků. Díky tomu budeme mít v~budoucnu snadnou cestu k~využití zbývajících receptů bez fotografií, ať už pro účely strojového učení nebo i~zobrazení uživateli, pokud by větší nabídka receptů výrazně převážila nevýhodu absence ilustračních fotografií.

Také data k~ingrediencím budou vyžadovat významné čištění. V~datasetu Food.com Recipes and~Interactions máme k~dispozici přibližně $8\,000$ unikátních ingrediencí. Co nejvíce z~nich bychom chtěli nabídnout uživateli v~rámci našeptávače ve vyhledávání dle ingrediencí. Pro tento účel názvy ingrediencí převedeme do estetičtějšího formátu s~velkým počátečním písmenem. Po manuální kontrole seznamu ingrediencí ale narazíme na řadu slov, která se mezi suroviny dostala omylem vlivem chybného parsování jmen ingrediencí. Nebudeme zde uvádět kompletní výčet, typicky se ale jedná o~názvy jednotek měření nebo obecné fragmenty ingrediencí, které samy o~sobě žádnou ingredienci nepředstavují (např. samostatná slova \texttt{clove}, \texttt{seed}, \texttt{extract}, která by byla validní pouze v~kontextu typu \texttt{garlic clove}, \texttt{sesame seed} a~\texttt{vanilla extract}). Vzhledem k~velkému počtu ingrediencí navrhneme heuristiku čištění pomocí regulárních výrazů. Zaměříme se zejména na nejčastěji používané ingredience, které budou zobrazeny v~horní části našeptávače. Pro potřeby našeptávače nastavíme limit maximálního počtu slov ingredience a~to na hodnotu $3$. Pro mapování na entity ze znalostních grafů ale využijeme původní sadu ingrediencí bez omezení počtu slov.

S~normalizovanými ingrediencemi z~Food.com Recipes and~Interactions souvisí další problém --- nejsou přiřazeny ke všem receptům z~datasetu. Texty surovin sice využijeme z~extrahovaného JSON-LD, normalizované ingredience ale potřebujeme k~propojení s~informacemi z~grafů DBpedia a~Wikidata. Recepty bez normalizovaných ingrediencí tedy musíme projít a~pro každou jejich přísadu zkusit na základě prostého textu nalézt co nejbližší shodu s~některou z~normalizovaných ingrediencí. Pro zjednodušení budeme akceptovat pouze přesné shody, přestože nám tímto způsobem může část ingrediencí uniknout, neboť mohou být v~prostém textu uvedeny v~jiném pádě nebo čísle.

Dalším úkolem čisticí fáze bude normalizace JSON-LD reprezentace ingrediencí z~grafu DBpedia. Ve srovnání s~projektem Wikidata jsme zde ve výhodě, jelikož data obdržíme přímo v JSON-LD. Zároveň ale extrahujeme data pro více ingrediencí najednou a~každá z~nich má vlastní schéma, které se většinou plně neshoduje s~ostatními entitami. Při skupinové extrakci dat se ale musí vytvořit univerzální schéma, kterým lze vyjádřit všechny obsažené informace. Naším úkolem bude projít uložené kolekce ingrediencí a~pro každou ingredienci vytvořit minimální JSON-LD kontext, kterým ji lze popsat. Jednotlivé ingredience pak do databáze uložíme vždy s~vlastním JSON-LD kontextem. V~praxi se totiž často stává, že objevíme pod stejnou vlastností různé typy hodnot. Např. region původu ingredience může nést IRI příslušné entity z~grafu DBpedia, ale také prostý literál. Skutečný typ musí být řádně definován kontextem JSON-LD dokumentu. Proto není vhodné mít společný kontext pro všechny ingredience, neboť by u~některých vlastností existoval duplicitní popis použitých typů.

\subsection{Databázový model}

Pro uložení dat receptů a~ingrediencí zvolíme dokumentovou databázi Apache CouchDB, často označovanou zkráceně CouchDB. Nejbližší alternativou z~řad NoSQL dokumentových databází by byl systém MongoDB, který je stejně jako databáze CouchDB open-source. Pro naše potřeby by dobře fungoval libovolný z~těchto systémů, neboť chceme využít zejména konceptu dokumentových databází, které typicky nevyžadují striktní schéma a~umožňují tak kompaktní uložení různorodých dat ve formátu JSON. To je v~naší doméně cenná vlastnost, neboť plánujeme ukládat data poměrně komplexní struktury --- vezměme v~potaz například JSON-LD formát s~mnoha zanořenými objekty. Také extrahujeme data z~více zdrojů a~vyžadování zcela jednotného rozhraní by nám v~některých situacích zbytečně zkomplikovalo práci. Na základě jakého kritéria tedy rozhodujeme mezi variantami CouchDB a~MongoDB?

Databázi budeme pokládat pouze velmi jednoduché dotazy, totiž vyžádání dokumentu na základě jeho unikátního id. Pro práci s~databází preferujeme použití Node.js knihovny, což je splnitelné oběma databázovými systémy CouchDB i~MongoDB, neboť oba poskytují oficiální ovladač pro Node.js. K~obsluze složitějších vyhledávacích dotazů využijeme systém Apache Solr, který si bude držet vlastní podmnožinu dat z~dokumentů uložených v~databázi. Kombinací Apache CouchDB s~Apache Solr bychom sjednotili technologie z~dílny Apache Software Foundation. Zároveň by se nám v~budoucnu mohla hodit podpora současného čtení a~zápisu, kterou systém CouchDB nabízí \citep{mongodb-vs-couchdb}. Co se týče dat s~recepty či ingrediencemi, není potřeba vyžadovat, aby se uživateli vždy zobrazila verze se všemi aktualizacemi. Např. vybraný recept zůstává relevantní i~v~momentě, kdy zrovna současný nebo jiný uživatel přidává k~receptu nové hodnocení. Díky systému verzování dokumentů, který CouchDB implementuje, se nemusíme obávat o~dostupnost dat během jejich aktualizace.

V~systému CouchDB lze snadno vytvářet a~spravovat více databází neboli kolekcí dokumentů. Vytvoříme tedy samostatné databáze pro recepty a~ingredience, přičemž oba typy dokumentů v~sobě budou mít uložena strukturovaná data i~JSON-LD reprezentaci. Na rozdíl od tradičních relačních databází nemusíme předem definovat žádná schémata ukládaných dat.

\subsection{Indexy}

Pro uložení dokumentů do databáze CouchDB není potřeba specifikovat žádné schéma. Během konfigurace platformy Apache Solr pro vyhledávání se ovšem bez striktně definovaného schématu neobejdeme. Respektive abychom byli přesní, Solr dokáže na základě vložených dat odvodit schéma dynamicky, často ale nezvolí datový typ správně a~navíc je při výběru zbytečně generický. Představme si libovolný řetězec z~dokumentu receptu, například text jedné ingredience. Pokud Solr nenajde ve schématu žádnou definici typu u~vlastnosti ingredience, vytvoří pro ni dynamický index typu \texttt{string}. Na první pohled se zdá být vše v~pořádku, když si ale projdeme dostupné typy, najdeme mezi nimi i~podstatně specifičtější variantu \texttt{text\underline{{ }}en}. Typ anglického textu je pro naši aktuální lokalizaci ideální, neboť nabízí vestavěnou podporu skloňování a~časování anglických slov. V~praxi nám pomůže najít více relevantních výsledků. Uživatel může zadat například ingredienci \texttt{tomatoes}, Solr provede normalizaci vyhledávaného termínu a~vrátí recepty obsahující mezi surovinami nejen slova \texttt{tomatoes}, ale také \texttt{tomato}.

Prvním krokem pro práci s~platformou Solr je tedy kompletní návrh schématu pro dokumenty receptů. Pro vytvoření schématu využijeme skript operující nad REST~API poskytovaným systémem Solr, čímž proces automatizujeme a~usnadníme případnou migraci na jiné zařízení. Solr umožňuje data rozdělit do tzv.~jader, založíme tedy samostatné jádro pro dokumenty s~recepty a~vytvoříme infrastrukturu, která umožní snadné přidání nového jádra, pokud by v~budoucnu bylo potřeba. V~rámci současných požadavků aplikace by se mohlo zdát, že budeme vyhledávací schopnosti platformy Solr potřebovat i~pro dokumenty ingrediencí, konkrétně pro řešení našeptávače surovin. Seznam doporučených přísad by se totiž měl aktualizovat na základě vstupu uživatele. Po napsání každého nového znaku se musí zobrazit pouze ingredience, které odpovídají vyhledávanému výrazu. Přirozeně nebereme v~potaz velikost písmen a~vzhledem k~výhradně anglické lokalizaci se v~současnosti nezabýváme ani diakritikou. Využití systému Solr by nám pomohlo nabídnout více relevantních ingrediencí, neboť bychom vyhledávaný výraz mohli interpretovat jako anglický text a~poradit si tak s~jeho formátem v~různých tvarech a časech. Kvůli každému napsanému znaku bychom ale museli odeslat dotaz serveru hostícímu instanci Solr, což by generovalo poměrně významné zpoždění vlivem komunikace po síti. Spokojíme se tedy s~jednodušší architekturou našeptávače --- vyžádáme si všechny ingredience najednou a~zobrazení relevantních návrhů během psaní vyřešíme na straně klienta. Jak lze ale získat seznam všech unikátních ingrediencí, pokud máme v~Solr uloženy pouze dokumenty receptů? Odpovědí je fasetové vyhledávání.

\subsubsection{Fasetové vyhledávání}

Fasetové vyhledávání, označované také jako fasetová navigace, je způsob interakce, během které uživatel filtruje výsledky výběrem validních hodnot fasetového klasifikačního systému. Tento styl vyhledávání nevyžaduje hierarchické uspořádání nabízených možností, díky čemuž lze filtry přidávat i~odebírat v~libovolném pořadí. Navíc uživatel typicky předem zná počet výsledků, které se po aplikování daného filtru zobrazí \citep{faceted-search}. V~kontextu naší aplikace se fasetová navigace hodí pro jednotlivé vlastnosti vyhledávání, jakými jsou nejen ingredience, ale také klíčová slova, kategorie, typ kuchyně, čas přípravy či hodnocení. Každý z~těchto filtrů je zcela nezávislý, není tedy žádoucí vytvářet kolem nich hierarchii. Nedávalo by smysl zpřístupnit například vyhledávání dle ingrediencí až po výběru kategorie receptu. Na druhou stranu, výběr kategorií může ovlivnit (respektive omezit) nabídku ingrediencí ve fasetové navigaci a~stejně tak volba určitých ingrediencí může vyřadit některé kategorie receptů.

Platforma Solr poskytuje přímou podporu fasetové navigace nad libovolnými položkami dokumentů. Můžeme tedy například snadno specifikovat fasetové vyhledávání nad ingrediencemi, čímž získáme list unikátních jmen surovin, které se v~celé kolekci receptů vyskytují. Je zde ovšem jisté omezení. Pokud fasetové vyhledávání spustíme přímo nad ingrediencemi, které máme uloženy pod typem anglického textu, Solr nám vrátí pouze transformovaná jména ingrediencí, tak jak je má uložena pro své interní vyhledávání. Data tohoto formátu nejsou vhodná pro prezentaci uživateli, tudíž budeme potřebovat ke každému receptu přiřadit nový seznam ingrediencí určený výhradně pro fasetové vyhledávání. Na úrovni schématu definujeme typ fasetových ingrediencí jako prostý řetězec, nad kterým se neprovedou žádné transformace.

\subsubsection{Zvýraznění nalezených výrazů}

Dalším konceptem, se kterým se během vyhledávání pomocí Solr setkáme, bude tzv. \emph{highlighting} neboli zvýraznění vyhledaných výrazů. Využijeme jej při prezentaci surovin jakožto primárního filtru. U~libovolného receptu na vyhledávací obrazovce bude možné zobrazit kompletní seznam ingrediencí. Pro lepší přehlednost zvýrazníme aktuálně vyhledávané ingredience tučným písmem a~pro přesnou identifikaci těchto pojmů využijeme vestavěnou funkci od platformy Solr. Informace o~zvýrazněných ingrediencích doručíme na frontend aplikace spolu s~dokumenty receptů.

\subsubsection{Model receptu}

Schéma receptu navrhneme dle požadavků na vlastnosti, podle kterých potřebujeme recepty vyhledávat. Zároveň zde ale patří definice všech položek, které budeme na vyhledávací stránce zobrazovat. Teoreticky bychom mohli platformu Solr využít pouze na vyhledání identifikátorů receptů na základě zadaných indexů a~ostatní informace získat z~databáze CouchDB. Tento přístup by optimalizoval množství paměti využívané systémem Solr a~odstranil poměrně výraznou duplicitu dat. Na druhou stranu by do zpracování vyhledávacích dotazů zanesl větší komplexitu a~časovou prodlevu, která by vznikla nadbytečnou komunikací s~\,CouchDB. Pokud si budeme všechny potřebné informace pro vyhledávací stránku držet v~systému Solr, bude nám stačit zpracovat pouze jeden vyhledávací dotaz pro zobrazení jedné stránky receptů. Rychlé vyhledávání je klíčovou funkcionalitou naší aplikace, proto zde upřednostníme optimalizaci časové složitosti namísto paměťové.

Schéma dokumentu receptu uloženého v~Solr bude následující (jedná se pouze o~ilustrační schéma, kde typy odpovídají standardním typům dle specifikace Solr, nikoli běžně dostupným typům formátu JSON):

\begin{code}
{
    name: text_en,
    description: text_en,
    recipeCategory: text_en,
    ingredients: text_en,
    tags: text_en,
    rating: pfloat,
    reviewsCount: pint,
    stepsCount: pint,
    cookMinutes: pint,
    prepMinutes: pint,
    totalMinutes: pint,
    image: string,
    date: string,
    calories: pint,
    fat: pfloat,
    saturatedFat: pfloat,
    cholesterol: pfloat,
    sodium: pfloat,
    carbohydrate: pfloat,
    fiber: pfloat,
    sugar: pfloat,
    protein: pfloat,
}
\end{code}
%$

Ne všechny obsažené položky nutně využijeme v~první verzi naší prezentační vrstvy (například datum a~počet minut samotné přípravy či vaření pokrmu). Jejich zahrnutím ale usnadníme přidávání nových funkcí typu třídění výsledků na základě data publikace nebo vyřazení receptů, u kterých nestačí pouhá příprava ze syrových ingrediencí a~je potřeba počítat s~vařením. Větší výběr informací nám také umožní jednoduchou iteraci nad různými rozloženími uživatelských obrazovek, což je užitečné při hledání vhodného designu.

\subsection{Backend}

Jak bylo zmíněno v~úvodní části této kapitoly, serverová část aplikace bude sloužit jako poměrně jednoduchá mezivrstva pro zprostředkování komunikace mezi klientem, platformou Solr a~databází CouchDB. Základy architektury položíme na zvoleném frameworku Express.js pro Node.js, který umožňuje snadnou definici REST API a~delegování přijímaných požadavků na vlastní komponenty. Budeme podporovat $4$~endpointy pro HTTP GET požadavky, jmenovitě získání všech receptů nebo ingrediencí a~vyžádání $1$~receptu či ingredience dle jejich unikátního id. V těle odpovědi na libovolný dotaz budou figurovat data v~JSON formátu, naše rozhraní tedy můžeme označit jako JSON~API. Zároveň budeme pracovat s~query parametry, prostřednictvím kterých předáme serveru informaci o~požadovaných filtrech, počtu výsledků na $1$~stránku a~offsetu pro zajištění funkcionality stránkování.

V~závislosti na zvoleném endpointu se s~dotazem obrátíme na Solr nebo \,CouchDB. Pro oba systémy vytvoříme odpovídající modely spravovaných dokumentů. U~Solr nám bude stačit model pro recepty, pro CouchDB pak definujeme modely receptů i~surovin. Model bude mít za úkol vytvořit a~uchovat spojení s~úložištěm, přičemž vytvoření instance spojení zajistí příslušná komponenta factory s~využitím návrhového vzoru singleton. Díky němu se vyhneme opětovné inicializaci spojení. Dále bude model schopen vytvořit dotazy na základě informací z~query parametrů a~prostřednictvím navázaného spojení tyto dotazy vyhodnotit nad dokumenty v~Solr nebo CouchDB. Následně přijatá data převede do formátu očekávaného na frontendu aplikace (typicky zjednoduší výchozí schéma odpovědi a~relevantní data uloží s~menší mírou zanoření). Data budou poté ve formátu JSON a~s~příslušným stavovým kódem odeslána klientovi. 

\subsubsection{JSON API}

Zde soustředíme konkrétní podobu API pro získání JSON dokumentů s~recepty a~ingrediencemi. Prefixem potřebným pro složení kompletní URL adresy je název domény, v~našem vývojovém prostředí \texttt{localhost:5000}.

\paragraph{Získání všech dokumentů}\mbox{}\\

Dotaz na recepty využijeme v~kontextu vyhledávací obrazovky. Požadavek obohatíme o~query parametry nesoucí informace o~požadovaných ingrediencích, klíčových slovech, kategoriích a~dalších filtrech. Za normálních okolností si vyžádáme pouze omezené množství výsledků odpovídající počtu receptů na $1$~stránce. Tím získáme odpověď v~podstatně kratším čase, což se projeví dřívějším zahájením renderování výsledků vyhledávání. Chceme uživateli API umožnit nastavení vlastního limitu počtu výsledků, neboť se limit může na frontendu měnit nezávisle na zbytku aplikace. Navíc tím také usnadníme práci vývojářům, kteří by potřebovali z~naší aplikace extrahovat strukturovaná data, neboť by jim stačil výrazně menší počet požadavků. Nastavíme ovšem maximální limit na počet výsledků v~rámci $1$~požadavku, abychom zamezili situacím, kdy se uživatel API pokusí najednou získat všechna data z~naší aplikace.

Ingredience máme uloženy pouze v~databázi CouchDB, dotaz z~endpointu ingrediencí tedy bude směřovat právě tam. S~aktuálními požadavky aplikace si bez tohoto endpointu vystačíme, je ale pravděpodobné, že budeme v~blízké době implementovat encyklopedii všech ingrediencí po vzoru webové aplikace Food.com.

\begin{code}
/api/recipes
/api/ingredients
\end{code}
%$

\paragraph{Získání dokumentu dle id}\mbox{}\\

Požadavky směřující na následují endpointy jsou vyřizovány nalezením detailu receptu nebo ingredience v~databázi CouchDB:

\begin{code}
/api/recipes/:recipeId
/api/ingredients/:ingredientId
\end{code}
%$

\subsection{Frontend}

Návrh klientské vrstvy bude silně ovlivněn charakterem knihovny React, která je založena na designu tzv. \emph{komponent}. Ty mohou být modelovány objektově orientovaným způsobem jako třídy nebo funkcionálním jako exportované funkce přijímající $1$ parametr označovaný jako \emph{props}. Autoři dokumentace knihovny React doporučují v~nových projektech využívat primárně funkcionální komponenty a~pro práci se stavem a~životním cyklem komponenty využít tzv. \emph{Hooks}. Většina z~důležitých funkcí, které byly dříve implementovány pouze v~objektových komponentách, je již dostupná funkcionálním komponentám skrze Hooks a~podpora zbývajících funkcí je plánována \citep{class-or-functional}. Architekturu tedy založíme na doporučených funkcionálních komponentách.

\subsubsection{Hierarchie komponent}

Se znalostí rozložení jednotlivých uživatelských obrazovek navrhneme příslušný systém komponent. Můžeme postupovat směrem od komponent s~největším dosahem, které odpovídají jedné obrazovce, ale také opačným směrem od komponent reprezentujících jednu základní část našeho datového modelu, například obrázek receptu. Snažíme se dodržovat princip jedné odpovědnosti a jakmile komplexita některé z~komponent začne příliš stoupat, provedeme dekompozici a~část práce přesuneme do komponenty potomka. Častým postupem je předání odkazu na funkci komponentě potomka. Funkce je pak zavolána jako callback například po stisknutí tlačítka, které je ve správě vnořené komponenty. Vztah předka a potomka bychom mohli přirovnat k~jeho pojetí v modelu DOM, kde by komponenty odpovídaly jednotlivým elementům, které lze do sebe vnořit. Nejedná se o~koncept dědičnost z~objektového návrhu, kde třída potomka vystupuje jako specializace třídy předka, musí splňovat všechny vlastnosti předka a~volitelně poskytovat dodatečné vlastnosti a~funkce.

\paragraph{Kostra aplikace}\mbox{}\\

Kořenem našeho stromu komponent bude \texttt{BrowserRouter}, komponenta pocházející z~knihovny React Router. Ta bude obsahovat jedinou komponentu \texttt{App}, kterou dále rozdělíme na základní komponenty \texttt{Header}, \texttt{Routes} a~\texttt{Footer}. Komponenty \texttt{Header} a~\texttt{Footer} budou společné pro celou aplikaci, proto mohou být na úrovni komponenty \texttt{Routes}. Potomci komponenty \texttt{Routes} budou odpovídat jednotlivým obrazovkám, budeme tedy mít $3$ komponenty \texttt{Route} s~následujícími cestami provázanými s URL adresami:

\begin{code}
/recipes
/recipes/:recipeId
/ingredients/:ingredientId
\end{code}
%$

Části adres označené jako \texttt{:recipeId} a~\texttt{:ingredientId} značí proměnné, za které budou dosazeny identifikátory konkrétních receptů či ingrediencí. Přidáme ještě dodatečnou komponentu \texttt{Route} pro zpracování domovské stránky, tedy s~cestou: \texttt{/}. Tato komponenta bude v~pilotní verzi naší aplikace pouze přesměrovávat na adresu \texttt{/recipes}. Grafické znázornění základní komponentové struktury aplikace viz schéma \ref{obr02:react-app}.

\begin{figure}[h!]\centering
\includegraphics[width=140mm]{../img/react-app}
\caption{Rozhraní aplikace založené na konceptu knihovny React Router.}
\label{obr02:react-app}
\end{figure}

\paragraph{Komponenta vyhledávání receptů}\mbox{}\\

Postup dekompozice si ukážeme na zjednodušeném převodu obrazovky pro vyhledávání receptů na komponenty knihovny React. Obsah celé vyhledávací obrazovky budeme reprezentovat komponentou \texttt{Recipes}. Ta bude zajišťovat mapování filtrů a~současné stránky prohlížení na query parametry URL adresy. Dále bude komunikovat s~REST API, konkrétně s~endpointem pro vyhledávání receptů \texttt{/api/recipes}. Výsledky receptů si vyžádá prostřednictvím asynchronního GET požadavku a~jakmile data obdrží, předá je specializovaným komponentám pro zajištění renderování. Těmito komponentami rozumíme potomky komponenty \texttt{Recipes}.

Jak bylo zmíněno v~začátku této sekce, potomci jsou vnořené komponenty, kterým lze předat data prostřednictvím parametru \texttt{props}. Příkladem vnořené komponenty bude \texttt{RecipesGrid}, která bude dále distribuovat informace o~receptech svým potomkům \texttt{RecipeCard}. Také komponenta \texttt{RecipeCard} bude složena z~menších komponent, konkrétně \texttt{RecipeCardContent}, \texttt{RecipeCardActions} a~\texttt{RecipeCardCollapse}. Rozhraní vyhledávaného receptu bude odpovídat schématu definovanému pro Solr, pojmenujme jej \texttt{SimpleRecipe} pro kontrast s~podrobnějším rozhraním receptu na stránce detailu. Diagram zachycující výše uvedené komponenty včetně parametrů \texttt{props} a~vzájemných vztahů je ilustrován obrázkem \ref{obr02:recipes-component}.

\begin{figure}[h!]\centering
\includegraphics[width=145mm]{../img/recipes-component}
\caption{Diagram znázorňující dekompozici komponenty \texttt{Recipes}.}
\label{obr02:recipes-component}
\end{figure}

Komponenta \texttt{RecipeCardContent} se postará o~zobrazení obrázku, názvu receptu s~úryvkem popisu, hodnocení včetně počtu recenzí a~také času přípravy. Komponenta \texttt{RecipeCardActions} odpovídá spodnímu panelu karty receptu, se kterým lze interagovat. Je zde umístěné tlačítko pro zobrazení receptu a~také tlačítko pro rozbalení seznamu ingrediencí. Samotný seznam ingrediencí již není ve správě komponenty \texttt{RecipeCardActions}. Ta se jen stará o~jeho aktivaci, která je zajištěna zavoláním příslušné funkce předané jako parametr z~rodičovské komponenty \texttt{RecipeCard}. Proměnnou indikující stav rozbalení sekce s~ingrediencemi vlastní komponenta \texttt{RecipeCard}. Při zavolání funkce se tento stav změní a~komponenta \texttt{RecipeCard} předá novou hodnotu komponentě \texttt{RecipeCardCollapse}. Ta na základě aktualizované hodnoty zobrazí nebo skryje seznam ingrediencí.

Finální dekompozice komponenty \texttt{Recipes} bude samozřejmě o~něco složitější, neboť si kromě karet receptů musíme poradit také se zobrazením vyhledávacích filtrů, stránkováním a~nadpisem indikujícím aktuální stav vyhledávání. Rovněž budeme chtít uživatele informovat o~úspěšně přidaných nebo odstraněných filtrech prostřednictvím krátké notifikace. Během implementace uplatníme výše popsané principy dekompozice a~také se pokusíme v~co největší míře zapojit již hotové komponenty poskytované knihovnou Material UI.

\subsection{Alternativní správa dat}

Při návrhu architektury ukládání a~dotazování se nad daty receptů a~ingrediencí bychom se mohli vydat cestou konstrukce vlastního znalostního grafu. Museli bychom definovat ontologii, pomocí které bychom byli schopni kompletně popsat svou doménu receptů i~surovin. Sestrojený graf bychom nahráli do tzv.~\emph{RDF triplestore}, což je typ grafové databáze specializované na jednoduchá RDF tvrzení ve formátu trojic: \emph{subjekt}, \emph{predikát} a~\emph{objekt}. Vrcholy grafu jsou reprezentovány entitami subjektů a~objektů, orientované hrany mezi nimi modelujeme pomocí predikátů. Nad uloženými daty se lze následně dotazovat pomocí jazyka SPARQL, se kterým jsme se setkali při extrakci dat ze znalostních grafů DBpedia a~Wikidata.

\subsubsection{Projekt Linked Recipes}

Výše uvedený přístup byl otestován v~rámci souvisejícího projektu Linked \,Recipes\footnote{https://github.com/lhotanok/LinkedRecipes}, který vznikl jako semestrální práce v~předmětu Data na Webu vedeném RNDr.~Jakubem~Klímkem,~Ph.D. v~roce $2022$. Součástí projektu bylo získání dat s~recepty ze $3$~různých zdrojů, dále jejich převedení do RDF formátu v~libovolné serializaci, nalezení linků mezi datasety navzájem i~s~externími grafy a~na závěr vytvoření jednoduché aplikace, která všechna data propojila a~umožnila nad nimi spouštět SPARQL dotazy. Jako statické zdroje dat byly vybrány datasety Food.com Recipes and~Interactions a~BBC Recipes. Ty byly obohaceny o~dataset vygenerovaný z~webové aplikace Allrecipes a~jako zástupci externích grafů byly zvoleny projekty DBpedia a~Wikidata. Je zde tedy velký překryv s~datovými sadami využívanými v~naší aplikaci pro vyhledávání receptů.

Aplikace Linked Recipes je napsána v~jazyce Java a~pro obsluhu HTTP požadavků využívá koncept tzv.~servletů, které generují odpovídající HTML dokumenty, často na základě komunikace s~databází. Aplikace používá RDF triplestore jako úložiště dat, konkrétně implementaci frameworku Eclipse RDF4J. Data ve formátu RDF jsou při spuštění aplikace nahrána do paměti prostřednictvím úložiště \emph{MemoryStore}, které je dle dokumentace třídy vhodné pro dataset s~méně než $100\,000$ tvrzeními. Alternativně lze použít úložiště provázané se SPARQL endpointem, který přijímá dotazy přes HTTP.

Aplikace si díky RDF triplestore a~SPARQL dotazům dokáže poměrně snadno poradit s~vyhledáním receptů dle ingrediencí, včetně zákazu vybraných ingrediencí. Také zvládá filtrovat recepty na základě povoleného rozmezí nutričních hodnot. Dalším v~aplikaci demonstrovaným příkazem je sestavení grafu ingrediencí s~využitím dat z~grafu Wikidata. Aplikace má k~dispozici seznam entit z~Wikidata, které se namapovaly na ingredience datasetů s~recepty. Přímo za běhu pak získává odkazy na obrázky k~daným ingrediencím a~vkládá je do HTML dokumentu. Nepotřebuje si tedy data k~ingrediencím z~Wikidata ukládat do interní databáze, čímž se zjednoduší architektura aplikace. Nevýhodou je ale podstatně delší čas potřebný k~získání a~zobrazení informací.

Pokud bychom se rozhodli pro stejný návrh aplikace jako v~projektu Linked Recipes, nahradili bychom vrstvy CouchDB a~Solr úložištěm RDF triplestore. To by v~sobě mělo uložena všechna data stejně jako CouchDB a~zároveň by řešilo vyhledávací dotazy namísto platformy Solr. Rychlost vyhledávání by byla závislá pouze na optimalizaci SPARQL dotazů. Narazili bychom na problém, pokud bychom se nespokojili s~vyhledáváním dle přesné shody a~vyžadovali chytřejší řešení po vzoru Solr a~jeho vyhledávání v~anglickém textu. Dále bychom se museli vypořádat s~požadavkem fasetového vyhledávání, pro které máme v~Solr přímou podporu na straně serveru. Fasetové vyhledávání se obvykle nepoužívá v~přímé kombinaci s~RDF modelem z~důvodu rizika, že by příkazy potřebné k~jeho podpoře byly příliš komplexní a~jejich výkon nedostačující \citep{sparql-facet}. Sehnat dostatečně výkonné a~dobře zdokumentované řešení by tedy bylo podstatně složitější v porovnání s~využitím osvědčeného Solr nebo jeho alternativ v~podobě ElasticSearch\footnote{https://www.elastic.co/guide/en/app-search/current/facets-guide.html} či Sphinx\footnote{http://sphinxsearch.com/blog/2013/06/21/faceted-search-with-sphinx/}.
%%% Fiktivní kapitola s ukázkami tabulek, obrázků a kódu

\chapter{Implementace návrhu}

Před zahájením vývoje aplikace je potřeba nainstalovat všechny potřebné nástroje a nakonfigurovat vývojové prostředí. Nejdůležitější nástroje, které vyžadují globální instalaci, jsou následující:

\begin{itemize}
    \item Node.js
    \item Apache CouchDB
    \item Apache Solr
    \item Silk Workbench
    \item Apify CLI
\end{itemize}

V řešení využijeme také řadu knihoven, které budeme instalovat pouze pro daný projekt pomocí výchozího správce balíčků npm pro Node.js. Tyto knihovny vždy uvedeme na seznamu závislostí projektu, takže se nainstalují snadno prostřednictvím příkazu \texttt{npm install}.

Co se týče výběru programovacích jazyků, přípravu dat vyřešíme pomocí vzájemně nezávislých skriptů psaných v jazyce JavaScript. Samotnou aplikaci včetně serverové a klientské vrstvy již napíšeme jazykem TypeScript, který je potřeba následně transpilovat do JavaScriptu. Tento dodatečný krok přidává komplexitu při spouštění kódu, proto jej vynecháme u jednoduchých skriptů připravujících dokumenty pro databázi a Solr. Zároveň ale přináší typovou kontrolu, kterou velmi oceníme v komplexnější aplikaci a to zejména při práci s externími knihovnami, jejichž rozhraní není vždy perfektně zdokumentováno.

\section{Vývojové prostředí}

Pro vývoj aplikace zvolíme editor Visual Studio Code s rozšířeními pro jazyky JavaScript a TypeScript. 

\section{Zpracování vstupních dat}


\section{Databáze Apache CouchDB}


\section{Vyhledávání pomocí Apache Solr}


\section{Middleware}


\section{Single-page aplikace}
\chapter{Testování}

\section{Uživatelské testování}

Uživatelské rozhraní a~celkový dojem z~aplikace otestujeme pomocí rozšířené metody System Usability Scale, označované zkráceně jako SUS \citep{sus-test}. Tuto metodu představil John Brooke již v~roce $1986$, stále je ale velmi aktuální, o~čemž svědčí například její zmínka ve více než $1300$ publikacích.

Účastníkům testování nejprve představíme základní funkce, tedy vyhledávání receptů pomocí různých filtrů, zobrazení detailu receptu s~odkazy na rozšířené informace o~ingrediencích a~obsah stránky s~detailem ingredience. Následně jim uložíme několik úkolů, které je přimějí seznámit se s~aplikací více do hloubky. Po splnění úkolů necháme uživatele vyplnit standardizovaný dotazník s~$10$ otázkami a~na základě odpovědí vypočteme koeficient úspěšnosti.

Výsledné skóre dotazníku se pohybuje v rozmezí od $0$ do $100$, nejedná se ale o~procentuální stupnici, jak by se mohlo na první pohled zdát. Důležitá je hranice $68$ bodů. Pokud je překročena, použitelnost systému je hodnocena jako nadprůměrná. Skóre lze dále normalizovat pro přesnější interpretaci výsledků, viz~obrázek \ref{obr04:sus-score-system}. Nám vyšlo na základě odpovědí $4$ respondentů průměrné skóre $93,125$, což lze považovat za velmi dobrý výsledek. Detaily výpočtu a~konkrétní odpovědi z~dotazníku jsou umístěny v~příloze.

\begin{figure}[h!]\centering
\includegraphics[width=140mm]{../img/sus-score-system}
\caption{Interpretace skóre při testování metodou System Usability Score. Adaptováno ze schématu SUS pro UX \citep{sus-adobe}.}
\label{obr04:sus-score-system}
\end{figure}

\subsection{Zadané úkoly}

Uživatelům zadáme následující úlohy, které simulují běžné využívání aplikace:

\begin{enumerate}
    \item Najděte recepty, které obsahují kuřecí prsa, rajčata a~parmazán a~je možné je připravit do $3/4$ hodiny. Poté vyhledávání změňte na recepty, které lze připravit do $1$~a~půl hodiny.
    \item Najděte nejrychlejší veganské recepty vhodné pro přípravu snídaně.
    \item Zadejte libovolné filtry a~projděte výsledky vyhledávání na první a~druhé straně. Vyberte $3$~recepty, které vás nejvíce zaujaly a~zběžně projděte jejich obsah včetně postupu přípravy.
    \item Najděte recepty z~mexické kuchyně s vysokým obsahem bílkovin a~nízkým obsahem sacharidů. Bez otevření detailů zjistěte, z~jakých ingrediencí jsou vyrobeny.
    \item Prozkoumejte ingredience na stránce vybraného receptu a~najděte suroviny ze stejných kategorií.
\end{enumerate}

\subsection{Zpětná vazba}

Nezávisle na SUS dotazníku jsme požádali testující také o~celkovou zpětnou vazbu a~nápady na vylepšení v~rámci dalšího vývoje. Celkový dojem z~používání aplikace byl pozitivní, obdrželi jsme ale řadu připomínek a~návrhů, díky kterým by práce s~aplikací byla pohodlnější a~jednodušší. Výběr nejdůležitějších z~nich, které budou vyřešeny přednostně:

\begin{enumerate}
    \item Při otevřeném receptu nebo ingredienci by měla být v~horní části stránky možnost navigace, aby uživatel nemusel využívat tlačítko \texttt{Zpět} ve webovém prohlížeči, které je zbytečně daleko a~ruší dojem single-page aplikace. Aktuální navigace v~podobě \emph{breadcrumbs} na stránce receptu je pro tento scénář nedostačující, neboť při otevření kategorie \texttt{Recipes} jsou ztraceny původní vyhledávací filtry.
    \item Názvy některých ingrediencí z~našeptávače pro vyhledávání obsahují překlepy nebo gramatické chyby.
    \item Řada kategorií na stránkách ingrediencí je příliš odborná a~do aplikace tohoto typu se spíše nehodí.
    \item Měly by být k~dispozici další jazykové lokalizace. Spoustu anglických termínů ingrediencí a~klíčových slov běžný uživatel nezná, pokud není rodilý mluvčí.
\end{enumerate}

Kromě závěrečného testování probíhaly také průběžné konzultace, na základě kterých byla do aplikace přidána funkce cachování, dále byly opraveny chyby s~nesprávně namapovanými ingrediencemi a~byl zkorigován vzhled aplikace.

\section{Výkonnostní testování}

Zátěžové testy provedeme v~aplikaci pracující přibližně s~$91\,000$ recepty a~té\-měř $800$ ingrediencemi s~rozšiřujícími informacemi. Polovina dokumentů s~recepty pochází z~aplikace Food.com, druhá polovina z~Allrecipes. Při vyhledávání se zobrazuje maximálně $24$ výsledků na jedné straně pro optimalizaci renderování.

Všechny součásti aplikace včetně databáze CouchDB, vyhledávací platformy Solr a~serverové i~klientské části běží na jednom zařízení se stejnou IP adresou. Nemůžeme tedy zcela simulovat reálnou situaci, kdy se do zpracování vyhledávacích dotazů vloží komunikace se vzdáleným zařízením přes HTTP. Všechny API~požadavky totiž vyřizujeme lokálně.

Testovacím zařízením bude notebook Huawei se $4$-jádrovým ~procesorem Intel Core $i5$ $10$. generace a~$16$GB pamětí RAM. Na zařízení je nainstalován operační systém Windows $10$ a hlavní webové prohlížeče, na které cílíme, tedy Google Chrome, Mozilla Firefox a~Microsoft Edge. Testy výkonu provedeme pouze v prohlížeči Google Chrome, u~ostatních dvou pouze zkontrolujeme správné zobrazení aplikace. Využijeme vestavěný nástroj pro diagnostiku výkonu dostupný pod záložkou \texttt{Performance insights}. Pro účely testování vytvoříme produkční verzi aplikace pomocí příkazu \texttt{npm\,run\,build} a~následného \texttt{serve\,-s\,build}.

\subsection{Diagnostika výkonu}

Podíváme se na rychlost získávání dat ze serverové části aplikace, renderování obsahu HTML prostřednictvím modelu DOM, ale také na vizuální stabilitu jednotlivých elementů. Kromě těchto údajů diagnostika odhalila některé déle běžící funkce na hlavním vlákně. U~nich je doporučena obecná optimalizace v~podobě snížení času výpočtu na hlavním vlákně pod hranici $50\,$ms. Celkově jsou naměřené časy v~normě vzhledem k~nefunkčním požadavkům aplikace z~kapitoly o~analýze.

\subsubsection{Vyhledávání receptů}

Testy ukázaly, že obsah DOM pro obrazovku s~vyhledáváním receptů je bez zadání jakýchkoli filtrů načten průměrně za $0,5\,$s. Z toho přibližně $180\,$ms trvá získání receptů ze Solr. Vyžádáme si sice pouze $24$ receptů, chceme ale znát celkový počet pro aktuální vyhledávací dotaz. Kvůli tomu trvá zpracování dotazu na všechny recepty delší dobu než vyhledávání s~omezujícím filtrem nebo více filtry.

Z~filtrovacích dotazů jsme nejprve testovali ingredienci \texttt{Cheese}, pro kterou je k~dispozici téměř $23\,000$ receptů. Zpracování tohoto dotazu trvalo průměrně $115\,$ms, tedy přibližně o třetinu kratší dobu oproti vyhledávání všech receptů. Se zadáváním dalších filtrů klesá čas potřebný pro získání dat až na průměrných $25\,$ms pro dotaz filtrující právě $1$~recept. Tento klesající trend je pro naši aplikaci příznivý, neboť cílíme na specifické filtrování pomocí většího množství kritérií. Nicméně pokud bychom datovou sadu s~recepty výrazně rozšiřovali, museli bychom optimalizovat dotaz na všechny výsledky bez zadaných filtrů. Jeho zpracování by bez optimalizace trvalo příliš dlouho, což by bylo krajně nevhodné, neboť se jedná o~dotaz aktivovaný automaticky při otevření domovské stránky a~tedy první interakci uživatele s~aplikací.

\subsubsection{Detail receptu}

Získání dat receptu na základě id trvá průměrně $10\,$ms. Načtení DOM pak $0,24\,$s. Z~následného vykreslení obvykle připadne nejdelší čas (kolem $0,85\,$s) na popis receptu, pokud je text delší.

\subsubsection{Detail ingredience}

Načtení dokumentu ingredience z~backendu trvá stejně jako u~receptu přibližně $10\,$ms a~načtení DOM asi $0,3\,$s. Maximální čas vykreslení je opět spotřebován renderováním popisu (pro ingredienci \texttt{Beef} se jednalo o~$0,72\,$s).

\subsubsection{Vizuální stabilita}

Celkově by dle výsledků testování bylo potřeba zapracovat na tzv.~\emph{Cumulative Layout Shift} (CLS). Jedná se o~metriku zaměřenou na vizuální stabilitu rozložení jednotlivých elementů v HTML dokumentu. Typicky je ohrožena asynchronními požadavky, na základě kterých se data načítají postupně a~způsobují tak přeskládávání elementů. Je tedy vhodné ještě před načtením dat vytvořit aspoň základní rozložení v~očekávané velikosti, aby se následným přesunům zamezilo nebo byl zmírněn jejich dosah. Často totiž předem nevíme, v~jaké velikosti data obdržíme a~nemůžeme tak přesně odhadnout rozměry příslušného elementu. Nicméně už od skóre $0,25$ je potřeba cílit na zlepšení \citep{cls-metric} a~diagnostický nástroj u~některých z~našich elementů zachytil podstatně vyšší skóre $0,65$. Jedná se například o~sloupec s~nutričními hodnotami a~ingrediencemi na stránce detailu receptu nebo o~společnou patičku stránky.

\chapter*{Závěr}
\addcontentsline{toc}{chapter}{Závěr}

Cílem této bakalářské práce bylo vytvořit aplikaci pro vyhledávání receptů, která agreguje data z~různých zdrojů s~využitím principu propojených dat. Demonstrovali jsme přínosy a~úskalí statických datasetů i~vlastní extrakce dat. Dále jsme zapojili informace z~otevřených znalostních grafů, které nám umožnily nejen rozšířit data o~jednotlivých ingrediencích, ale také objevit nová spojení mezi přísadami a~tranzitivně i~recepty.

V~prvotní fázi vývoje byly zadefinovány funkční i~nefunkční požadavky aplikace, na základě kterých byla navržena architektura a~uživatelské rozhraní. Výběr požadavků byl veden analýzou existujících webových aplikací pro vyhledávání receptů. Naše práce si totiž stanovila cíl propojit užitečné funkce z~osvědčených aplikací a~nabídnout přidanou hodnotu díky jejich zkombinování v~rámci jedné aplikace. Volba klíčových technologií se ukázala jako vhodná pro naplnění požadavků.

Práce se soustředila pouze na získání a~prezentaci existujících dat receptů a~ingrediencí. Přidávání nových dokumentů prostřednictvím uživatelského rozhraní aplikace bude otázkou dalšího vývoje spolu s~registrací uživatelů, ukládání oblíbených receptů, nákupního seznamu a~dalších souvisejících funkcí.

Zkorigování dat z~více zdrojů vyžaduje poměrně velké množství manuální práce, neboť mohou obsahovat řadu odchylek od standardního formátu. Pokud bychom ale měli k~dispozici dostatečnou časovou dotaci a~výpočetní prostředky, aplikace by mohla vyniknout právě velkým množstvím dostupných dat. S~rostoucí datovou sadou bychom byli schopni nabídnout rozmanitější filtrování výsledků nebo také personalizované návrhy receptů.

Aktuální řešení je připraveno pro nasazení a~vývoj dalších funkcionalit. Postup extrakce a~ukládání nových dokumentů je v~co největší míře automatizován pomocí vlastních skriptů, stejně jako definice schématu pro vyhledávání na základě indexů.

%%% Seznam použité literatury
\include{literatura}

%%% Obrázky v bakalářské práci
%%% (pokud jich je malé množství, obvykle není třeba seznam uvádět)
\listoffigures

%%% Tabulky v bakalářské práci (opět nemusí být nutné uvádět)
%%% U matematických prací může být lepší přemístit seznam tabulek na začátek práce.
% \listoftables

%%% Přílohy k bakalářské práci, existují-li. Každá příloha musí být alespoň jednou
%%% odkazována z vlastního textu práce. Přílohy se číslují.
%%%
%%% Do tištěné verze se spíše hodí přílohy, které lze číst a prohlížet (dodatečné
%%% tabulky a grafy, různé textové doplňky, ukázky výstupů z počítačových programů,
%%% apod.). Do elektronické verze se hodí přílohy, které budou spíše používány
%%% v elektronické podobě než čteny (zdrojové kódy programů, datové soubory,
%%% interaktivní grafy apod.). Elektronické přílohy se nahrávají do SISu a lze
%%% je také do práce vložit na CD/DVD. Povolené formáty souborů specifikuje
%%% opatření rektora č. 72/2017.
\appendix
\chapter{Přílohy}

\section{Instalace a spuštění aplikace}

Konfiguraci aplikace rozdělíme do $3$ fází:

\begin{enumerate}
    \item Příprava dat
    \item Persistence dat
    \item Spuštění aplikace
\end{enumerate}

Předpokládá se nainstalovaný runtime Node.js pro vývoj v~jazyce JavaScript s výchozím správcem balíčků npm. Systémy Apache Solr a~Silk Workbench byly zprovozněny pomocí Dockeru jako samostatné kontejnery, je ale možné zvolit jiný způsob instalace. Ke zpracování dat receptů a~ingrediencí jsou přiloženy shellové skripty pro systémy Windows i~Linux. Počínaje konfiguračními skripty \texttt{build.bat}, respektive \texttt{build.sh}, z~adresáře \texttt{data} se rekurzivně volají skripty z~vnořených adresářů. Dále jsou připraveny skripty \texttt{run.bat} a~\texttt{run.sh} pro spuštění extrakce, konverze dat a~jejich následné nahrání do databáze. Jednotlivé JavaScriptové programy lze spouštět samostatně pomocí příkazu \texttt{node\,script.js}. Často budeme potřebovat spustit jen některé fáze, zejména extrakce bude vzhledem k~objemu dat výjimečnou záležitostí. Shellové skripty tak slouží hlavně k~definici správného pořadí jednotlivých skriptů. Konstanty potřebné pro konfiguraci jsou obvykle soustředěny v~souborech \texttt{constants.js} a~\texttt{config.js}. Jejich hodnoty lze přizpůsobit dle potřeby, například omezit počet receptů, které se mají extrahovat z~webové aplikace pomocí Apify actora.

\subsection{Příprava dat}

Předpokládáme následující nainstalované nástroje:

\begin{itemize}
    \item Python
    \item Apify CLI\footnote{https://www.npmjs.com/package/apify-cli/v/0.1.2}
    \item Silk Workbench\footnote{https://hub.docker.com/r/silkframework/silk-workbench}
\end{itemize}

V~rámci fáze extrakce a~předzpracování dat je potřeba stáhnout datovou sadu \emph{Food.com Recipes and~Interactions}\footnote{https://www.kaggle.com/datasets/shuyangli94/food-com-recipes-and-user-interactions} z~Kaggle. Soubory \texttt{RAW\underline{{ }}recipes.csv} a~\texttt{PP\underline{{ }}recipes.csv} patří do adresáře \texttt{data/resources/foodcom/recipes}. Dále je potřeba umístit \texttt{ingr\underline{{ }}map.csv} do \texttt{data/resources/foodcom/ingredients}. Následně jsou připraveny skripty pro konverzi a~extrakci dat z~webových aplikací Food.com a~Allrecipes. Pro spuštění \texttt{run-venv.bat}, eventuálně \texttt{run-venv.sh} uvnitř adresáře \texttt{data/resources/foodcom/ingredients} je potřeba mít aktivováno virtuální prostředí pro vývoj v~jazyce Python s~nainstalovanou knihovnou pickle.

Ve všech adresářích obsahujících soubor \texttt{package.json} je potřeba použít příkaz \texttt{npm\,install}. Inicializace Apify actorů v~adresářích \texttt{food-com-scraper} a~\texttt{allrecipes-scraper} se provede standardními příkazy \texttt{npm\,install} a~dále \texttt{apify\,init}, který vytvoří potřebné podadresáře včetně \texttt{apify\underline{{ }}storage}. Zejména adresáře s~actory jsou zcela nezávislé na kontextu zbytku aplikace, aby bylo co nejsnadnější je migrovat na cloudovou platformu Apify, pro kterou jsou primárně určeny.

Dále je potřeba zkopírovat vygenerované datasety ve formátu Turtle do aplikace Silk Workbench a~vytvořit pro ně úlohy linkování s~otevřenými znalostními grafy DBpedia a~Wikidata. V~podadresářích \texttt{rdf-data} uvnitř adresářů \texttt{foodcom} a~\texttt{allrecipes} je uložen konfigurační XML soubor vygenerovaný ze Silk Workbench. Pomocí něj je možné úlohu spustit z~příkazové řádky, naše řešení ale tohoto přístupu nevyužívá. Datasety s~propojenými entitami jsou očekávány v~adresářích \texttt{rdf-data} ve formátu N-Triples.

\subsection{Persistence dat}

Předpokládáme nainstalovanou instanci systému Apache CouchDB s~platnými přihlašovacími údaji a~přístupem do administrace skrze webovou aplikaci Fauxton. Dále instalaci platformy Apache Solr s~vytvořeným jádrem \texttt{recipes}.

Extrahovaná a~vyčištěná data jsou uložena do databáze CouchDB. Příslušné skripty a~konfigurační soubory najdeme v~adresáři \texttt{data/database} a~jsou volány pomocí automatizovaného skriptu run z~kořenového adresáře \texttt{data}. Heslo pro přihlášení do databáze je potřeba uložit jako proměnnou prostředí pod názvem \texttt{COUCHDB\underline{{ }}PASSWORD} nebo změnit nastavení v souboru \texttt{config.js}.

Následuje vytvoření indexů a~odpovídajících dokumentů receptů pro platformu Solr. Související skripty jsou v~adresáři \texttt{data/solr}. Dokumenty receptů jsou extrahovány přímo z~databáze CouchDB, což by mělo usnadnit situaci, kdy by instance CouchDB a~Solr běžely na různých zařízeních.

\subsection{Spuštění aplikace}

Kód aplikace je soustředěn v~adresáři \texttt{app}, který je dále rozdělen na \texttt{frontend} a~\texttt{backend}. K~inicializace obou částí aplikace stačí nainstalovat závislosti dle \texttt{package.json} pomocí \texttt{npm\,install}. Pro následné spuštění serverové části aplikace je potřeba spustit z~umístění \texttt{backend} příkazy \texttt{npm\,run\,build} pro automatickou transpilaci z~TypeScriptu do JavaScriptu a~\texttt{npm\,start} k~samotnému spuštění aplikace. Pro úspěšný start musí být k~dispozici CouchDB i~Solr na adresách specifikovaných v~příslušných souborech \texttt{config.js}. Single-page aplikaci lze spustit pomocí \texttt{npm\,start} z~adresáře \texttt{frontend}. Veškeré změny v~kódu backendu i~frontendu se po uložení automaticky projeví v~aplikaci běžící ve webovém prohlížeči.

\pagebreak
\section{Hledání linků mezi datasety ingrediencí}

\begin{figure}[h!]\centering
\includegraphics[height=220mm]{../img/silk-workbench}
\caption{Grafické znázornění úlohy linkování v aplikaci Silk Workbench.}
\label{obr0a:silk-workbench}
\end{figure}

\section{Výsledky testování SUS}

Pro otestování UX designu naší aplikace jsme využili nástroj System Usability Scale neboli SUS \citep{sus-test}. Testující dostali $5$ jednoduchých úkolů uvedených v kapitole o testování a vyzkoušeli si skrze ně běžnou práci s aplikací. Posledním úkolem bylo zodpovědět $10$ standardizovaných otázek dle SUS dotazníku na škále od $1$ do $5$, kde $1$ představuje \uv{silně nesouhlasím} a $5$ naopak \uv{silně souhlasím}. 

\subsection{Formát dotazníku}
Otázky jsme dle vzoru z anglického zdroje zformulovali následovně \citep{sus-adobe}:

\begin{enumerate}
    \item Myslím, že bych rád aplikaci využíval pravidelně.
    \item Aplikace mi připadá zbytečně složitá.
    \item S aplikací se mi pracovalo snadno.
    \item Myslím, že bych potřeboval technickou podporu.
    \item Myslím, že funkce byly v aplikaci dobře integrovány.
    \item Zdá se mi, že aplikace byla příliš nekonzistentní.
    \item Domnívám se, že většina lidí by se naučila aplikaci používat velmi rychle.
    \item Ovládání aplikace mi připadalo dost těžkopádné.
    \item Při používání aplikace jsem se cítil velmi jistý.
    \item Musel jsem se naučit hodně věcí, než jsem mohl začít aplikaci používat.
\end{enumerate}

\subsection{Postup výpočtu skóre}

Algoritmus výpočtu finálního skóre je následující \citep{sus-adobe}:

\begin{enumerate}
    \item Zvol proměnnou $X$ jako součet celkového skóre otázek s lichým pořadím snížený o hodnotu $5$.
    \item Zvol proměnnou $Y$ jako výsledek celkového skóre otázek se sudým pořadím odečtený od $25$.
    \item Sečti hodnoty proměnných $X$, $Y$ a výsledek vynásob koeficientem $2,5$.
\end{enumerate}

$X := \sum x_i$ pro $i \in \{1,3,5,7,9\}$

$Y := \sum y_j$ pro $j \in \{2,4,6,8,10\}$

$Z := ( \,X + Y) \, \ast 2,5$

\subsection{Výpočet skóre}

Odpovědi vyjádříme pomocí uspořádaných $10$-tic v pořadí od $1$. do $10$. odpovědi a~s~indexem odpovídajícím pořadí respondenta.

$v_1 := (\,4,1,4,1,5,2,5,1,5,1)\,$

$X_1 := (\,4 + 4 + 5 + 5 + 5) - 5 = 18\,$

$Y_1 := 25 - (\,1 + 1 + 2 + 1 + 1) = 19\,$

$Z := ( \,18 + 25) \, \ast 2,5 = 92,5$

\openright
\end{document}
