\chapter*{Úvod}
\addcontentsline{toc}{chapter}{Úvod}

Vyhledávání relevantního obsahu je spolu s elektronickou komunikací jednou z~klíčových funkcí internetu. S~rostoucím množstvím dostupných informací se filtrování nalezených výsledků stává stále obtížnějším. Tvůrci webových stránek se často zaměřují spíše na uživatelsky přívětivé interaktivní rozhraní, zatímco optimalizace strojového vyhledávání jde stranou. Pro webové vyhledávače, jmenovitě např. Google, Bing nebo Yahoo, je pak náročné analyzovat obsah těchto stránek po sémantické stránce a~tedy vyhodnotit, zda obsahují užitečné informace k~zodpovězení dotazu uživatele.

V~reakci na tuto problematiku vznikl tzv. \emph{Sémantický Web} neboli Web dat jakožto rozšíření původního Webu dokumentů, tak jak jej známe z platformy \emph{World Wide Web}. Sémantický Web lze vnímat jako globální databázi, nad kterou se lze pomocí speciálního jazyka \emph{SPARQL} dotazovat podobně jako nad tradičními databázovými systémy. Data jsou poskytována v~různých serializacích formátu RDF a~mohou být přímo vnořena do HTML dokumentů nebo zpřístupněna v~samostatných souborech. Tato strukturovaná data nazýváme \emph{propojená} (v~originále \emph{Linked Data}). Umožňují snadnější hledání souvislostí mezi entitami z~různých zdrojů na základě společných slovníků neboli ontologií \citep{semantic-web}.

V posledních letech termín Sémantický Web ustupuje do pozadí a~často je místo něj zmiňován tzv. \emph{znalostní graf} (anglicky \emph{Knowledge Graph}). Začátky fenoménu znalostních grafů bychom mohli datovat do roku 2012, kdy společnost Google představila svůj znalostní graf pro vyhledávání obsahu na webu. K~technologii znalostních grafů se brzy poté přihlásily další velké společnosti včetně firem Microsoft, IBM, Facebook, LinkedIn, Amazon, eBay, Airbnb nebo Uber. Grafový model totiž oproti tradičnímu relačnímu modelu nabízí flexibilnější správu dat z~oblasti sociálních sítí, dopravních spojení, bibliografických citací a~řady dalších odvětví. Výše zmíněné příklady znalostních grafů všechny spadají do kategorie komerčních znalostních grafů, které jsou určeny pro interní využití v~rámci dané firmy. Protikladem jim jsou otevřené znalostní grafy poskytující data k~volnému využití všem uživatelům internetu. Nejvýznamnějšími představiteli otevřených znalostních grafů jsou aktuálně DBpedia, Wikidata, Freebase a YAGO \citep{kg-book}. První dva zmíněné projekty si představíme v~této práci a~integrujeme je s~aplikací na vyhledávání receptů.

Oblast gastronomie je rozvěž vhodným kandidátem k~zapojení do sítě znalostních grafů a~propojených dat. Pro tvůrce webových aplikací je poměrně jednoduché publikovat obsah svých stránek ve formátu strukturovaných dat. Vhodným způsobem je např. vložení RDF reprezentace daných entit (receptů, uživatelů, recenzí) ve formátu JSON-LD\footnote{Koncovka \emph{LD} v~názvu JSON-LD odkazuje na pojem Linked Data.} přímo do hlavičky jednotlivých HTML dokumentů. V~takovém případě je žádoucí použít existující ontologie raději než definovat vlastní, byť by mohly být lépe strukturované a uzpůsobené dané doméně. Využití standardizovaných slovníků usnadňuje webovým vyhledávačům interpretaci stránky a je větší šance, že se aplikace dostane na vyšší příčky vyhledávaných výsledků.

Cílem této bakalářské práce je prozkoumat možnosti využití otevřených dat v~doméně receptů, propojit je s~daty publikovanými na různých webových stránkách shromažďujících recepty a~prezentovat tyto výsledky uživateli ve formě vlastní webové aplikace. Zároveň v~rámci této aplikace poskytnout užitečné možnosti filtrování agregovaných výsledků včetně fasetového vyhledávání. Proces sběru, konverze a~uložení dat by měl být co nejvíce automatizovaný a~snadno zreprodukovatelný. Práce se nevěnuje přidávání nových receptů prostřednictvím uživatelského rozhraní. Existujících webové stránky totiž obsahují velké množství dat, které lze díky bohaté historii v~podobě hodnocení a~recenzí lépe filtrovat. Navíc by bylo potřeba se vypořádat s~automatickou kalkulací nutričních hodnot receptu z~obsažených surovin, přičemž ne všechny ingredience dokážeme automaticky identifikovat a~získat jejich nutriční hodnoty. V~budoucnu by funkce nahrávání nových receptů měla být přidána spolu s~více lokalizacemi aplikace, registrací uživatelů a~celkovou personalizací obsahu pro přihlášené uživatele.

\section*{Volba tématu}
\setcounter{tocdepth}{1}

Příprava jídla je tématem každodenního života a~na webových stránkách, které se této oblasti věnují, má velmi silnou komunitu. Většina z~nás se chystání domácích pokrmů z~ekonomických důvodů nevyhne, takže se hodí mít po ruce sadu receptů pro inspiraci. Typicky máme na recepty různé požadavky - někdo preferuje rychlejší postup, jiný se dívá po ceně ingrediencí nebo nutričních hodnotách. Občas dostaneme chuť na recept z řecké nebo italské kuchyně a~jindy zkrátka chceme experimentovat a~najít recept kombinující našich 5 oblíbených surovin. Některé ingredience z~receptu nám mohou být neznámé, nebo si jen podle názvu nejsme jistí, zda máme na mysli tu správnou. V~takovém případě musíme stránku s~receptem opustit a~dodatečné informace k~ingredienci vyhledat jinde, pokud na ně aplikace přímo neodkazuje. Zde je příležitost zapojit otevřená data a~namapovat názvy ingrediencí na jejich odpovídající entity ve znalostních grafech. Data pak můžeme začlenit do aplikace a~nabídnout uživateli informace nad rámec samotného receptu, např. popisy a glykemické hodnoty surovin, ilustrační obrázky a podobně. Také můžeme identifikovat ingredience a~tranzitivně recepty ze stejných kategorií. Oproti původní datové sadě tak vytvoříme nové vazby a~poskytneme uživateli rozmanitější filtrování výsledků.

Doména receptů navíc poskytuje spoustu prostoru pro zajímavá rozšíření se zapojením moderních technologií. Uplatnění by zde našlo například počítačové vidění s~rozpoznáváním obrázků. S~dostatečně velkou databází bychom díky němu mohli analyzovat fotografii hotového pokrmu a nalézt příslušný recept. Usnadnili bychom tak uživateli práci v~situacích jako je návštěva restaurace, při které návštěvníkovi zachutnalo servírované jídlo a~chtěl by si jej později připravit v domácích podmínkách. Uživatelé by také mohli ocenit výhody populárního \emph{full-text} vyhledávání. Snadno by s~ním objevili recepty na základě klíčových slov v~popisku receptu, postupu či recenzích. V~komerční sféře by se nabízelo propojení s online supermarkety, konkrétně zrychlení nákupu pomocí vyhledávání surovin k~vybranému receptu. S~tímto konceptem již na svých stránkách pracuje firma rohlik.cz, nabídka receptů a možnosti filtrování jsou ale omezené. Nepochybně by se hodilo integrovat také doporučovací systém pro ještě snadnější nalezení relevantních výsledků. Aplikace má velký prostor pro škálování objemu dat, přičemž datasety mohou být následně použity jako podklad pro strojové učení.