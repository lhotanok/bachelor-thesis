\chapter*{Úvod}
\addcontentsline{toc}{chapter}{Úvod}

Vyhledávání relevantního obsahu je spolu s elektronickou komunikací jednou z~klíčových funkcí internetu. S~rostoucím množstvím dostupných informací se filtrování nalezených výsledků stává stále obtížnějším. Tvůrci webových stránek se často zaměřují spíše na uživatelsky přívětivé interaktivní rozhraní, zatímco optimalizace strojového vyhledávání jde stranou. Pro webové vyhledávače, jmenovitě např. Google, Bing nebo Yahoo, je pak náročné analyzovat obsah těchto stránek po sémantické stránce a~tedy vyhodnotit, zda obsahují užitečné informace k~zodpovězení zadaného dotazu.

V~reakci na tuto problematiku vznikl tzv. \emph{Sémantický Web} neboli Web dat jakožto rozšíření původního Webu dokumentů, tak jak jej známe z platformy \emph{World Wide Web}. Sémantický Web lze vnímat jako globální databázi, nad kterou se lze pomocí speciálního jazyka \emph{SPARQL} dotazovat podobně jako nad tradičními databázovými systémy. Data jsou poskytována v různých serializacích formátu RDF a mohou být přímo vnořena do HTML dokumentů nebo zpřístupněna v~samostatných souborech. Pojem Sémantický Web úzce souvisí s~propojenými daty (v~originále \emph{Linked Data}), která jsou publikována právě v RDF formátu a~umožňují snadnější hledání souvislostí mezi entitami z~různých zdrojů na základě společných slovníků neboli ontologií, viz \citet{semantic-web}.

V posledních letech termín Sémantický Web ustupuje do pozadí a často je místo něj zmiňován tzv. znalostní graf (anglicky \emph{Knowledge Graph}). Začátky fenoménu znalostních grafů bychom mohli datovat do roku 2012, kdy společnost Google představila svůj znalostní graf pro vyhledávání na webu.

